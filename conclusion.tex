\chapter{Conclusion}
\section{Summary of Findings}
By using linear analysis, bifurcation theory, and numerical simulations, we have performed a thorough stability analysis for reaction-diffusion mechanisms with incorporated gene expression time delays. In this dissertation, kinetic equations associated with both Schnakenberg and Gierer-Meinhardt reaction kinetics were considered, and gene expression delays modelled as both a fixed parameter, and as a continuous distribution. The results in this dissertation help facilitate our understanding of the effect that time delay has on Turing pattern formation, and highlight the importance of their consideration in biological patterning events.

By combining Fourier analysis and numerical simulations, we presented a strong linearly increasing relationship, between fixed time delay, and the time-lag until onset of patterning for the Schnakenberg model. Through numerical simulations, a linearly increasing relationship was also shown between time delay and time-to-pattern for both GM variants considered. These results suggest that the pathological nature of Turing pattern formation is a more general property of the reaction-diffusion mechanisms, and is not limited to a specific choice in kinetics. A systematic review of the robustness of pattern formation to varying initial conditions was conducted. Motivated by the biology considered in \cite{krausemixed}, Dirichlet boundary conditions were implemented for the activator dynamics. It was found that, for the Schnakenberg model, although the \textit{type} of pattern seen changes with these variations, the increase in lag until onset of patterning as a result of time delay is robust and consistent.

For the Schnakenberg kinetics, where fixed gene expression delays were motivated by ligand internalisation models, we have noted that extensive ligand internalisation acts to expand the Turing space. This is in contrast to the results concluded in \cite{yigaffneyli}. This effect was displayed through the use of bifurcation diagrams, and was confirmed through numerical solutions.
Motivated by the stability analysis of spike solutions of the GM model in \cite{fadai1,fadai2}, we demonstrated the importance of the positioning of time-delayed terms within a reaction-diffusion mechanism. This in turn highlights the importance of the biological processes we are considering, and shows that a complete understanding of the molecular processes are needed before Turing mechanisms can be applied to real patterning events. Linear and bifurcation analysis showed that for the GM model, increasing of time delay can either act to expand or contract the Turing space. We also note that the expansion and contraction of all the Turing spaces considered was solely dependent on the spatially homogeneous models. This yields an interesting question as to whether this is a more general mechanism by which time delay can affect the parameter space exhibiting Turing instabilities, or whether it is a specific attribute of the Schnakenberg and GM kinetics.

Finally, driven by the inherent stochasticity of the molecular processes underpinning gene expression \cite{raj,elowitz,mcadams,paulsson}, gene expression time delays were modelled as both a symmetric and skewed Gaussian distribution. Through linear analysis, and verified by numerical simulations, it was presented that the distribution used does not matter. Namely, the pattern formation process of the Schnakenberg model seems to be dependent on the mean delay of the distribution used, irrespective of standard deviation or skew, and thus can affectively be modelled as purely a fixed delay.

\section{Future Work}

Our findings, that a distributed representation of time delay does not alleviate the increased timescales of patterning events caused with a fixed delay, calls into question how relevant and applicable Turing mechanisms are for describing biological patterning events. Despite these results, empirical evidence suggest that such Turing instabilities do exist to explain biological phenomena \cite{yigaffneyli,molecular,miura,miura2,sick}. Our research does not close the door to applying Turing's models, but in fact yields an abundance of new and unanswered questions. We first note the extreme simplicity of the reaction-diffusion models we consider in this dissertaton, in contrast to the complexity of the biological processes, whose behaviour we attempt to capture. In \cite{mainigeneral}, work was done to develop Turing conditions for a system describing the interaction of $n$ morphogens, for any $n\geq2$. However, as far as the authors of this dissertation are aware, there is no systematic study of how time delay may affect pattern formation events for a reaction-diffusion mechanism with greater than two morphogens. Therefore, one potentially important avenue for further research would be to investigate the effect of time delay on Turing mechanisms encapsulating a larger number of reactants. This would aid in improving the possible applicability and similarity of Turing's models to the more intricate biological dynamics.

Throughout this disssertation, we considered two different types of kinetics, namely those derived from Schnakenberg and Gierer-Meinhardt kinetic reactions. Our results indicate some general attributes that are common to both sets of kinetics. The first being a linearly increasing relationship between incorporated time delay and time until onset of patterning. The second being that the effect of time delay on the exhibited Turing space is only dependent on the spatially homogeneous model, irrespective of whether the Turing space is growing or shrinking. A clear extension to this observation would be to explore these effects for different reaction-diffusion systems that can exhibit Turing patterns. Typical models that could be examined include the Gray-Scott \cite{grayscott} or Thomas \cite{murray} models.

We note that many simplifying assumptions were made in the models that we considered. Representing time delay as a continuous distribution is a novel field of interest, and thus has not previously been explored in depth. The use of other forms of distribution, such as the gamma or exponential distributions could be considered, in order to verify our findings that, when onset of patterning is being considered, the only relevant modelling parameter required is the mean delay. We also only considered the problem on a one-dimensional stationary spatial domain. Previous research has been conducted on higher dimensional spatial domains, and growing domains, with incorporated fixed delay. Although we hypothesise that our results with a distributed delay will be consistent across variations in the spatial domain considered, there is room to explore these possibilities. Finally, we note that due to numerical limitations when using \textit{chebfun} to find roots of the transcendental characteristic equations, the linear theory could only be tested for small time delays. We found that, for these small time delays, the linear theory generally provided a good approximation to the time-to-pattern, and all conclusions from the linear theory were able to be verified through full numerical solutions. This however yields an interesting question of, what the limitations of the linear theory are, and for how large of a timescale can the linear theory still be applied. Further work could therefore be done to solve the characteristic equations derived in this dissertation for larger time delay values, and examine whether the linear theory still provides good approximations to the model behaviour.
