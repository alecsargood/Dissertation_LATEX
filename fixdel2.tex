\chapter{Fixed Delay Model}

In this Chapter, we first revisit the analysis of the LI model through a more careful review of the linear theory presented in \cite{yigaffneyli,jiang}, with the aim of analytically

The authors in \cite{gaffmonk} numerically demonstrated that as the fixed time delay is increased in the LI model, the time taken for patterns to emerge also increases. In this Chapter, we first revisit the analysis of the LI model through a more careful review of the linear theory presented in \cite{yigaffneyli,jiang}, with the aim of analytically showing a linear dependance between a fixed delay $\tau$ and the time taken for pattern formation. This is not a result that has yet been formalised in the literature. We also show that, an increasing time delay in the LI model, where the time-delayed terms are placed only in the activator's dynamics, can act to stabilise the Turing instability region. This results in a larger parameter space where Turing instabilities may be found. We verify these results through numerical simulations. Through full numerical simulations, we also consider the effects of variations in both initial and boundary conditions on pattern formation, and find that the relationship between fixed time delay and time-to-pattern is robust under these variations.

\section{Linear Analysis}
As defined in \eqref{fixed}, the equations we study for the LI model are
\begin{equation}\label{fixed2}
  \begin{split}
  \frac{\partial u}{\partial t}&=\frac{\epsilon^2}{L^2}\frac{\partial^2u}{\partial x^2}+a-u-2u^2v+3\hat{u}^2\hat{v}\\
  \frac{\partial v}{\partial t}&=\frac{1}{L^2}\frac{\partial^2v}{\partial x^2}+b-u^2v,
\end{split}
\end{equation}
with no flux boundary conditions, and where $u=u(x,t)$, $v=v(x,t)$ and $\hat{u}$, $\hat{v}$ are evaluated at some delay $\tau$, so that $\hat{u}=u(x,t-\tau)$ and $\hat{v}=v(x,t-\tau)$. Following the methodology in \cite{yigaffneyli}, we take a small perturbation about the steady-state $u(x,t)=u_\star+\delta\xi(x,t)$ and $v(x,t)=v_\star+\delta\eta(x,t)$, where $|\delta|\ll 1$. Taylor expanding up to $O(\delta)$ about the steady-state, the linearised dynamics of \eqref{fixed2} are then given by
\begin{equation}\label{linfixed}
  \begin{split}
\frac{\partial\xi}{\partial t}&=\frac{\epsilon^2}{L^2}\frac{\partial^2\xi}{\partial x^2}-\xi-4u_\star v_\star\xi+6u_\star v_\star\hat{\xi}-2u_\star^2\eta+3u_\star^2\hat{\eta},\\
\frac{\partial\eta}{\partial t}&=\frac{1}{L^2}\frac{\partial^2\eta}{\partial x^2}-2u_\star v_\star\xi-u_\star^2\eta,
\end{split}
\end{equation}
with $\hat{\xi}=\xi(x,t-\tau)$ and $\hat{\eta}=\eta(x,t-\tau)$. Substituting into \eqref{linfixed} an ansatz of the form $\begin{pmatrix}\xi\\\eta\end{pmatrix}=\begin{pmatrix}\xi_0e^{-\lambda_k t}\cos(k\pi x)\\ \eta_0e^{-\lambda_k t}\cos(k\pi x)\end{pmatrix}$, we obtain the characteristic equation, $\mathcal{D}_k=0$, given by
\begin{equation}\label{characfix}
\mathcal{D}_k=\lambda_k^2+\alpha_k\lambda_k+\beta_k+(\gamma_k\lambda_k+\delta_k)e^{-\lambda_k\tau}=0,
\end{equation}
where the coefficients are given as,\footnote{We note the coefficient of $\beta_k$ differs from that of \cite{yigaffneyli} due to a typgraphical error in the cited paper.}
\begin{align}
\alpha_k&=\left(\frac{\epsilon^2}{L^2}+\frac{1}{L^2}\right)k^2\pi^2+u_\star^2+4u_\star v_\star+1,\\
\beta_k&=\left(\frac{1}{L^2}\pi^2k^2+u_\star^2\right)\left(\frac{\epsilon^2}{L^2}\pi^2k^2+4u_\star v_\star+1\right)-4u_\star^3v_\star,\\
\gamma_k&=-6u_\star v_\star,\\
\delta_k&=-{6}{L^2}u_\star v_\star k^2\pi^2.
\end{align}
This characteristic equation can be used to determine the parameter sets $(a,b,\epsilon^2,L,\tau)$ in which a Turing instability occurs, and hence where we expect pattern formation. From \eqref{perturbgrow}, the perturbation grows like $e^{\lambda_k t}\cos(k\pi x)$, and so if there exists a $k\neq0$ for a given $(a,b,\epsilon^2,L,\tau)$ such that $\Re(\max_k(\lambda_k))>0$, we expect pattern formation. Figure \ref{fig:dispfixed} shows $\max_k(\Re(\lambda_k))$ plotted against $\tau$, for multiple given $(a,b)$ parameter sets. The complex roots for $\lambda_k$ of the characteristic equation were found using the \emph{roots} command of the MATLAB package Chebfun \cite{chebfun}. These plots were produced by varying $k\in\mathbb{Z}$ over $[0,50]$ for a given $\tau$, and for each $k$, the roots of \eqref{characfix} were computed. The maximum over the $k$ of the $\Re(\lambda_k)$ was then taken. This was repeated for time delay varied over $\tau\in[0,1]$ at regular intervals of $0.1$. We do not consider a $k$ larger than $50$ as full numerical solutions for the parameter values used tended towards patterns with four `spikes', so we do expect large wave numbers to be excited.

\begin{figure}[H]
    \centering
    \begin{subfigure}[b]{0.45\textwidth}
        \centering
        \includegraphics[width=7cm,height = 5.5cm]{disp1.png}
        \caption{$(a,b)=(0.4,0.4)$. $\max_k(\Re(\lambda_k))<0 \quad \text{for all }\tau\in[0,1]$. Linear theory predicts no pattern formation for all $\tau\in[0,1]$. }
        \label{}
    \end{subfigure}
    \hfill
    \begin{subfigure}[b]{0.45\textwidth}
        \centering
        \includegraphics[width=7cm,height = 5.5cm]{disp2.png}
        \caption{$(a,b)=(0.1,0.9)$. $\max_k(\Re(\lambda_k))>0 \quad \text{for all }\tau\in[0,1]$. Linear theory predicts pattern formation for all $\tau\in[0, 1]$.}
        \label{}
    \end{subfigure}
    \caption{Characteristic equaton \eqref{characfix} solved and $\max_k(\Re{\lambda_k})$ plotted against $\tau\in[0,1]$ for two different parameter sets. $\epsilon^2=0.001$ and $L^2=9/2$.}
    \label{fig:dispfixed}
\end{figure}
\chapter{Fixed Delay Model}

In this Chapter, we first revisit the analysis of the LI model through a more careful review of the linear theory presented in \cite{yigaffneyli,jiang}, with the aim of analytically

The authors in \cite{gaffmonk} numerically demonstrated that as the fixed time delay is increased in the LI model, the time taken for patterns to emerge also increases. In this Chapter, we first revisit the analysis of the LI model through a more careful review of the linear theory presented in \cite{yigaffneyli,jiang}, with the aim of analytically showing a linear dependance between a fixed delay $\tau$ and the time taken for pattern formation. This is not a result that has yet been formalised in the literature. We also show that, an increasing time delay in the LI model, where the time-delayed terms are placed only in the activator's dynamics, can act to stabilise the Turing instability region. This results in a larger parameter space where Turing instabilities may be found. We verify these results through numerical simulations. Through full numerical simulations, we also consider the effects of variations in both initial and boundary conditions on pattern formation, and find that the relationship between fixed time delay and time-to-pattern is robust under these variations.

\section{Linear Analysis}
As defined in \eqref{fixed}, the equations we study for the LI model are
\begin{equation}\label{fixed2}
  \begin{split}
  \frac{\partial u}{\partial t}&=\frac{\epsilon^2}{L^2}\frac{\partial^2u}{\partial x^2}+a-u-2u^2v+3\hat{u}^2\hat{v}\\
  \frac{\partial v}{\partial t}&=\frac{1}{L^2}\frac{\partial^2v}{\partial x^2}+b-u^2v,
\end{split}
\end{equation}
with no flux boundary conditions, and where $u=u(x,t)$, $v=v(x,t)$ and $\hat{u}$, $\hat{v}$ are evaluated at some delay $\tau$, so that $\hat{u}=u(x,t-\tau)$ and $\hat{v}=v(x,t-\tau)$. Following the methodology in \cite{yigaffneyli}, we take a small perturbation about the steady-state $u(x,t)=u_\star+\delta\xi(x,t)$ and $v(x,t)=v_\star+\delta\eta(x,t)$, where $|\delta|\ll 1$. Taylor expanding up to $O(\delta)$ about the steady-state, the linearised dynamics of \eqref{fixed2} are then given by
\begin{equation}\label{linfixed}
  \begin{split}
\frac{\partial\xi}{\partial t}&=\frac{\epsilon^2}{L^2}\frac{\partial^2\xi}{\partial x^2}-\xi-4u_\star v_\star\xi+6u_\star v_\star\hat{\xi}-2u_\star^2\eta+3u_\star^2\hat{\eta},\\
\frac{\partial\eta}{\partial t}&=\frac{1}{L^2}\frac{\partial^2\eta}{\partial x^2}-2u_\star v_\star\xi-u_\star^2\eta,
\end{split}
\end{equation}
with $\hat{\xi}=\xi(x,t-\tau)$ and $\hat{\eta}=\eta(x,t-\tau)$. Substituting into \eqref{linfixed} an ansatz of the form $\begin{pmatrix}\xi\\\eta\end{pmatrix}=\begin{pmatrix}\xi_0e^{-\lambda_k t}\cos(k\pi x)\\ \eta_0e^{-\lambda_k t}\cos(k\pi x)\end{pmatrix}$, we obtain the characteristic equation, $\mathcal{D}_k=0$, given by
\begin{equation}\label{characfix}
\mathcal{D}_k=\lambda_k^2+\alpha_k\lambda_k+\beta_k+(\gamma_k\lambda_k+\delta_k)e^{-\lambda_k\tau}=0,
\end{equation}
where the coefficients are given as,\footnote{We note the coefficient of $\beta_k$ differs from that of \cite{yigaffneyli} due to a typgraphical error in the cited paper.}
\begin{align}
\alpha_k&=\left(\frac{\epsilon^2}{L^2}+\frac{1}{L^2}\right)k^2\pi^2+u_\star^2+4u_\star v_\star+1,\\
\beta_k&=\left(\frac{1}{L^2}\pi^2k^2+u_\star^2\right)\left(\frac{\epsilon^2}{L^2}\pi^2k^2+4u_\star v_\star+1\right)-4u_\star^3v_\star,\\
\gamma_k&=-6u_\star v_\star,\\
\delta_k&=-{6}{L^2}u_\star v_\star k^2\pi^2.
\end{align}
This characteristic equation can be used to determine the parameter sets $(a,b,\epsilon^2,L,\tau)$ in which a Turing instability occurs, and hence where we expect pattern formation. From \eqref{perturbgrow}, the perturbation grows like $e^{\lambda_k t}\cos(k\pi x)$, and so if there exists a $k\neq0$ for a given $(a,b,\epsilon^2,L,\tau)$ such that $\Re(\max_k(\lambda_k))>0$, we expect pattern formation. Figure \ref{fig:dispfixed} shows $\max_k(\Re(\lambda_k))$ plotted against $\tau$, for multiple given $(a,b)$ parameter sets. The complex roots for $\lambda_k$ of the characteristic equation were found using the \emph{roots} command of the MATLAB package Chebfun \cite{chebfun}. These plots were produced by varying $k\in\mathbb{Z}$ over $[0,50]$ for a given $\tau$, and for each $k$, the roots of \eqref{characfix} were computed. The maximum over the $k$ of the $\Re(\lambda_k)$ was then taken. This was repeated for time delay varied over $\tau\in[0,1]$ at regular intervals of $0.1$. We do not consider a $k$ larger than $50$ as full numerical solutions for the parameter values used tended towards patterns with four `spikes', so we do expect large wave numbers to be excited.

\begin{figure}[H]
    \centering
    \begin{subfigure}[b]{0.45\textwidth}
        \centering
        \includegraphics[width=7cm,height = 5.5cm]{disp1.png}
        \caption{$(a,b)=(0.4,0.4)$. $\max_k(\Re(\lambda_k))<0 \quad \text{for all }\tau\in[0,1]$. Linear theory predicts no pattern formation for all $\tau\in[0,1]$. }
        \label{}
    \end{subfigure}
    \hfill
    \begin{subfigure}[b]{0.45\textwidth}
        \centering
        \includegraphics[width=7cm,height = 5.5cm]{disp2.png}
        \caption{$(a,b)=(0.1,0.9)$. $\max_k(\Re(\lambda_k))>0 \quad \text{for all }\tau\in[0,1]$. Linear theory predicts pattern formation for all $\tau\in[0, 1]$.}
        \label{}
    \end{subfigure}
    \caption{Characteristic equaton \eqref{characfix} solved and $\max_k(\Re{\lambda_k})$ plotted against $\tau\in[0,1]$ for two different parameter sets. $\epsilon^2=0.001$ and $L^2=9/2$.}
    \label{fig:dispfixed}
\end{figure}
Figure \ref{fig:dispfixed} suggests that for $(a,b)=(0.4,0.4)$ and $\tau\in[0,1]$, no patterns will form, however, we expect to see pattern formation $(a,b)=(0.1,0.9)$. We also hypothesise that since $\max_k(\Re(\lambda_k))$ at $\tau=0$ is greater than at $\tau=1$, the time taken to pattern formation will be longer at $\tau=1$. This relationship between time-to-pattern and time delay is explored in more detail in section \ref{section:delaypatt}. Numerical results in figures \ref{fig:fixedsim1} and \ref{fig:fixedsim2} verify the findings from figure \ref{fig:dispfixed}, namely that pattern formation does not occur for $(a,b)=(0.4,0.4)$, but does for $(a,b)=(0.1,0.9)$, hence confirming predictions from the linear theory.

Since the Schnakenberg model has \textit{cross} reaction kinetics, as discussed in section \ref{section:background}, we have that when the concentration of the activator $u$ is high, the concentration of the inhibitor $v$ is low, and vise-versa \cite{murray}. The concentration gradients of the two morphogens $u$ and $v$ are thus effectively `out of phase', and so it is sufficient to consider just the numerical solution of the activator $u$. Throughout the disseration, the numerical solution of the activator $u$ is plotted. Unless otherwise stated, initial conditions $(u_0,v_0)$ are given as a random Gaussian perturbation from the homogeneous steady-state. Namely,
\begin{equation}\label{firstic}
\begin{pmatrix}u_0\\v_0\end{pmatrix}=\begin{pmatrix}u_\star(1+r)\\v_\star(1+r)\end{pmatrix},
\end{equation}
where $r$ is a random variable such that $r\sim\mathcal{N}\left(0,0.01^2\right)$. The notation $r\sim\mathcal{N}\left(\mu,\sigma_{\text{IC}}^2\right)$ denotes a Normally distributed random variable $r$ with mean $\mu$ and standard deviation of the initial perturbation $\sigma_{\text{IC}}$. A constant history function equal to the initial conditions is also used, unless otherwise stated.

\begin{figure}[H]
    \centering
    \begin{subfigure}[b]{0.45\textwidth}
        \centering
        \includegraphics[width=7cm,height = 5.5cm]{nopatt1.png}
        \caption{$\tau=0$. No pattern formation after $t=10^4$. }
        \label{}
    \end{subfigure}
    \hfill
    \begin{subfigure}[b]{0.45\textwidth}
        \centering
        \includegraphics[width=7cm,height = 5.5cm]{nopatt2.png}
        \caption{$\tau=1.$ No pattern formation after $t=10^4$.}
        \label{}
    \end{subfigure}
    \caption{Numerical simulations of \eqref{fixed2} showing no pattern formation with $(a,b)=(0.4,0.4)$, $\epsilon^2=0.001$ and $L^2=9/2$. Boundary conditions given by \eqref{neumannbc} and initial conditions by \eqref{firstic}.}
    \label{fig:fixedsim1}
\end{figure}1

\begin{figure}[H]
    \centering
    \begin{subfigure}[b]{0.45\textwidth}
        \centering
        \includegraphics[width=7cm,height = 5.5cm]{patt1.png}
        \caption{$\tau=0$. Distinct spikes formed at $t\approx7$ }
        \label{}
    \end{subfigure}
    \hfill
    \begin{subfigure}[b]{0.45\textwidth}
        \centering
        \includegraphics[width=7cm,height = 5.5cm]{patt2.png}
        \caption{$\tau=1$. Distinct spikes formed at $t\approx50$.}
        \label{}
    \end{subfigure}
    \caption{Numerical simulations of \eqref{fixed2} showing pattern formation for $(a,b)=(0.1,0.9)$, $\epsilon^2=0.001$ and $L^2=9/2$. Boundary conditions given by \eqref{neumannbc} and initial conditions by \eqref{firstic}.}
    \label{fig:fixedsim2}
\end{figure}

\subsection{Bifurcation Analysis}\label{section:fixedbif}
The Turing plot produced in figure \ref{fig:turingspace}, computed using the conditions in \eqref{cond1} and \eqref{cond2}, is a bifurcation diagram indicating regions of Turing instability. We note two separate curves which separate the parameter space into its distinct regions. These will be referred to as the `lines of stability'.
These two curves are indicated in figure \ref{fig:bif0}. The inner arc corresponds to the $(a,b)$ such that $\Re(\lambda_k)=0$ for the
spatially homogeneous characteristic equation, $\mathcal{D}_k=0$ when $k=0$. For $\tau=0$, this corresponds exactly to equating conditions $\eqref{cond1}$ to 0. The outer boundary is the points $(a,b)$ such that $\max_k\Re(\lambda_k)=0$ for the spatially inhomogeneous characteristic equation, $\mathcal{D}_k$ when $k\neq0$. For $\tau=0$, this is identical to equating to the conditions $\eqref{cond2}$ to 0. By letting $\lambda_k=x_k+iy_k$ for $x,y\in\mathbb{R}$, we split the characteristic equation $\mathcal{D}_k=0$ into its real and imaginary parts, $\mathcal{D}_k^{\Re}=0$ and $\mathcal{D}_k^{\Im}=0$, given by
\begin{align}\label{realfix}
\mathcal{D}_k^{\Re}&=x_k^2-y_k^2+\alpha_kx_k+\beta_k+e^{-x_k\tau}[\gamma_kx_k\cos(-y_k\tau)-\gamma_ky_k\sin(-y_k\tau)+\delta_k\cos(-y_k\tau)]=0,\\
\mathcal{D}_k^{\Im}&=2x_ky_k+\alpha_ky_k+e^{-x_k\tau}[\gamma_kx_k\sin(-y_k\tau)+\gamma_ky_k\cos(-y_k\tau)+\delta_k\sin(-y_k\tau)]=0.\label{complexfix}
\end{align}
By setting $\Re(\lambda_k)=x_k=0$ in equations \eqref{realfix} and \eqref{complexfix}, the real and imaginary parts of $\mathcal{D}_k$ can be simplified to
\begin{align}\label{realfixbif}
  \mathcal{D}_k^{\Re}&=-y_k^2\beta_k+[-\gamma_ky_k\sin(-y_k\tau)+\delta_k\cos(-y_k\tau)],\\
  \mathcal{D}_k^{\Im}&=\alpha_ky_k+[\gamma_ky_k\cos(-y_k\tau)+\delta_k\sin(-y_k\tau)].\label{complexfixbif}
\end{align}
For a fixed $\tau$ and $b$, the roots of \eqref{realfixbif} and \eqref{complexfixbif} (at $k=0$) can be found for $a$ and $\Im(\lambda_k)$.
Taking the $\max_k(a)$, a curve can be plotted in the $(a,b)$ parameter space for the outer boundary (and the inner arc) resulting in a bifurcation diagram of distinct regions where Turing instabilities can occur. We use a relatively large $L^2=9/2$ and so the bifurcation diagram computed in this manner for $\tau=0$ should be a good approximation to the Turing space plot produced in figure \ref{fig:turingspace}. Figure \ref{fig:bif0} shows the bifurcation plot produced in this manner for $\tau=0$ alongside the Turing space plot in figure \ref{fig:turingspace} for comparison.

\begin{figure}[H]
    \centering
    \begin{subfigure}[b]{0.47\textwidth}
        \centering
        \includegraphics[width=7cm,height = 5.5cm]{bif0.png}
        \caption{Stability lines for $\tau=0$ computed by solving characteristic equation with $\epsilon^2=0.001$, $L^2=9/2$.}
        \label{fig:bif0}
    \end{subfigure}
    \hfill
    \begin{subfigure}[b]{0.47\textwidth}
        \centering
        \includegraphics[width=7cm,height = 5.5cm]{turingspace.png}
        \caption{Turing space as in figure \ref{fig:turingspace} plotted from parameters $(a,b)$ satisfying conditions Turing conditions.}
        \label{}
    \end{subfigure}
    \caption{Comparison of Turing instability region for $\tau=0$ computed via characterstic equation(s) (\eqref{realfixbif} and \eqref{complexfixbif}) against Turing instability region computed via Turing conditions (\eqref{cond1} and \eqref{cond2}). Parameter space chosen as $(a,b)\in[0,1.4]\times[0,2]$.}
    \label{fig:tspace1}
\end{figure}

Figure \ref{fig:tspacetau} shows the stability curves computed for a varying $\tau\in\{0,0.5,1,1.5\}$. It can be seen that, whilst the outer boundary computed from the characteristic equation for the spatially inhomogeneous model stays the same, the inner arc computed from using the characteristic equation from the spatially homogeneous model shifts to the left, increasing the region of parameter space for which Turing instabilities can occur. This observation matches the results produced in \cite{william} for the spatially homogeneous LI model. This is also a similar result to that observed in \cite{fadai2} for the Gierer-Meinhardt model. We see that for the LI model where delay-terms are placed solely in the activator dynamics, time delay acts as a stabilising agent for pattern formation and increases the parameter space where Turing instabilities can occur.
\begin{figure}[H]
        \centering
        \includegraphics[width=12cm,height = 9cm]{tspacetau.png}
        \caption{Stability lines for $\tau\in\{0,0.5,1,1.5\}$ computed by solving \eqref{realfixbif} and \eqref{complexfixbif}. $\epsilon^2=0.001$, $L^2=9/2$.}
        \label{fig:tspacetau}
\end{figure}
We verify the results of figure \ref{fig:tspacetau} through numerical simulations. Three parameter points, $(a,b)=\{(0.12,0.5),(1.2,1.75),(1.2,1.85)\}$ are indicated in figure \ref{fig:tspacetau}. At $(a,b,\tau)=(0.12,0.5,0)$, linear theory suggests that there will be no pattern formation, but at $(a,b,\tau)=(0.12,0.5,1.5)$ there will be a Turing instability and thus patterns will form.
The parameter region in the bottom left of the parameter space is a delicate region that can exhibit both Turing and Hopf bifurcations, leading to complex spatio-temporal behaviours. This type of dynamics in reaction-diffusion systems has been studied more extensively in \cite{krausefixed,jiang}. Although the linear theory is unable to provide information about the more intricate nonlinear dynamics, it can predict the expected type of behaviour for certain parameter values. To show a change in behaviour as $\tau$ changes from $0$ to $1.5$, from a temporally oscillating solution, to one exhibiting a Turing pattern, we increase the diffusive ratio to $\epsilon^2=0.1$. This result can be seen in figure \ref{fig:testturing}. The linear theory also suggests that for all $\tau\in\{0,0.5,1,1.5\}$, pattern formation will occur for $(a,b)=(1.2,1.85)$, but not for $(a,b)=(1.2,1.75)$. Figures \ref{fig:testturing2} and \ref{fig:testturing3} show the results for numerical simulations at $(a,b)=\{(1.2,1.75),(1.2,1.85)\}$ for $\tau=\{0,0.5,1,1.5\}$.
\begin{figure}[h]
    \centering
    \begin{subfigure}[b]{0.45\textwidth}
        \centering
        \includegraphics[width=7cm,height=5cm]{toscill.png}
        \caption{$\tau=0$. Oscillations seen.}
        \label{}
    \end{subfigure}
    \hfill
    \begin{subfigure}[b]{0.45\textwidth}
        \centering
        \includegraphics[width=7cm,height=5cm]{tpattpred.png}
        \caption{$\tau=1.5$. Pattern formation seen.}
        \label{}
    \end{subfigure}
    \caption{Numerical simulations of \eqref{fixed2} produced with parameters $(a,b)=(0.12,0.5)$, for $\tau=0,1.5$. $\epsilon^2=0.1$ and $L^2=9/2$. Boundary conditions given by \eqref{neumannbc} and initial conditions by \eqref{firstic}. Linear theory in figure \ref{fig:tspacetau} suggests we see pattern formation at $\tau=1.5$ but not at $\tau=0$.}
    \label{fig:testturing}
\end{figure}

\begin{figure}[H]
    \centering
    \begin{subfigure}[b]{0.45\textwidth}
        \centering
        \includegraphics[width=7cm,height=4cm]{p3t0.png}
        \caption{$\tau=0$.}
        \label{}
    \end{subfigure}
    \hfill
    \begin{subfigure}[b]{0.45\textwidth}
        \centering
        \includegraphics[width=7cm,height=4cm]{p3t05.png}
        \caption{$\tau=0.5$}
        \label{}
    \end{subfigure}
    \hfill
    \begin{subfigure}[b]{0.45\textwidth}
        \centering
        \includegraphics[width=7cm,height=4cm]{p3t1.png}
        \caption{$\tau=1$}
        \label{}
    \end{subfigure}
    \hfill
    \begin{subfigure}[b]{0.45\textwidth}
        \centering
        \includegraphics[width=7cm,height=4cm]{p3t15.png}
        \caption{$\tau=1.5$.}
        \label{}
    \end{subfigure}
    \caption{Numerical simulations of \eqref{fixed2} for $(a,b)=(1.2,1.75)$. $\epsilon^2=0.001$ and $L^2=9/2$. Boundary conditions given by \eqref{neumannbc} and initial conditions by \eqref{firstic}. We see no pattern formation for $\tau\in\{0,0.5,1,1.5\}$ as suggested by linear theory, seen in figure \ref{fig:tspacetau}.}
    \label{fig:testturing2}
\end{figure}

\begin{figure}[H]
    \centering
    \begin{subfigure}[b]{0.45\textwidth}
        \centering
        \includegraphics[width=7cm,height=4cm]{p2t0.png}
        \caption{$\tau=0$.}
        \label{}
    \end{subfigure}
    \hfill
    \begin{subfigure}[b]{0.45\textwidth}
        \centering
        \includegraphics[width=7cm,height=4cm]{p2t05.png}
        \caption{$\tau=0.5$}
        \label{}
    \end{subfigure}
    \hfill
    \begin{subfigure}[b]{0.45\textwidth}
        \centering
        \includegraphics[width=7cm,height=4cm]{p2t1.png}
        \caption{$\tau=1$}
        \label{}
    \end{subfigure}
    \hfill
    \begin{subfigure}[b]{0.45\textwidth}
        \centering
        \includegraphics[width=7cm,height=4cm]{p2t15.png}
        \caption{$\tau=1.5$.}
        \label{}
    \end{subfigure}
    \caption{Numerical simulations of \eqref{fixed2} for $(a,b)=(1.2,1.85)$. $\epsilon^2=0.001$ and $L^2=9/2$. Boundary conditions given by \eqref{neumannbc} and initial conditions by \eqref{firstic}. We see pattern formation on an increasing time-scale for $\tau\in\{0,0.5,1,1.5\}$ as suggested by linear theory, seen in figure \ref{fig:tspacetau}. Increasing time-scale seen from changing $x$-axis limits.}
    \label{fig:testturing3}
\end{figure}

Figure \ref{fig:tspacetau} shows how the time delay affects the region of Turing instability, but it provides no information as to how $\max_k(\Re(\lambda_k))$ varies as $\tau$ increases over the $(a,b)$ parameter space. In figure \ref{fig:lambdavary} we plot a heatmap of $\max_k(\Re(\lambda_k))$ over the $(a,b)$ parameter space for varying $\tau\in\{0,0.5,1,1.5\}$. Overlayed onto these plots are contour lines corresponding to where $\Re(\lambda_0)=0$ and $\max_k(\Re(\lambda_k))=0$,
highlighting the Turing instability region. The $\max_k$ taken over $k\in[0,50]$ at regular discrete intervals of $1$.
\begin{figure}[H]
    \centering
    \begin{subfigure}[b]{0.45\textwidth}
        \centering
        \includegraphics[width=7cm,height=5cm]{tau0bif.png}
        \caption{$\tau=0$.}
        \label{}
    \end{subfigure}
    \hfill
    \begin{subfigure}[b]{0.45\textwidth}
        \centering
        \includegraphics[width=7cm,height=5cm]{tau05bif.png}
        \caption{$\tau=0.5$}
        \label{}
    \end{subfigure}
    \hfill
    \begin{subfigure}[b]{0.45\textwidth}
        \centering
        \includegraphics[width=7cm,height=5cm]{tau1bif.png}
        \caption{$\tau=1$}
        \label{}
    \end{subfigure}
    \hfill
    \begin{subfigure}[b]{0.45\textwidth}
        \centering
        \includegraphics[width=7cm,height=5cm]{tau15bif.png}
        \caption{$\tau=1.5$.}
        \label{}
    \end{subfigure}
    \caption{$\max_k(\Re(\lambda_k))$ computed over $(a,b)$ parameter space by solving \eqref{realfixbif} and \eqref{complexfixbif}, with $\epsilon^2=0.001$, $L^2=9/2$. As $\tau$ increases, $|\max_k(\Re(\lambda_k))|$ decreases. Contour lines for $\Re(\lambda_0)=0$ and $\max_k(\Re(\lambda_k))=0$ overlayed, indicated Turing instability region. $\max_k$ taken over $k\in[0, 50]$ at discrete intervals of $1$.}
    \label{fig:lambdavary}
\end{figure}
As $\tau$ increases, it can be seen that the absolute value $|\max_k(\Re((\lambda_k)))|$ also decreases. This suggests that for $(a,b)$ values within the Turing instability region, pattern formation will take longer to occur. It also suggests however that for $(a,b)$ such that $\max_k(\Re(\lambda_k))<0$, it will take a longer time for the eigenfunctions with modes $k\neq0$ to decay to a spatially homogeneous steady-state. We note this behaviour in figure \ref{fig:testturing2}, where it can be seen, by carefully considering the timescales, that the time taken for the initial perturbation to fully decay back to a spatially homogeneous steady-state increases as $\tau$ increases. Figure \ref{fig:fixbif2} shows analogous bifurcation diagrams as in figure \ref{fig:lambdavary}, but with $\epsilon^2=0.1$. We note that as the ratio of diffusion constants in the reaction-diffusion system, $\epsilon^2$, moves closer to $1$, the region of parameter space exhibiting Turing instability decreases. It can be observed however that altering $\epsilon^2$ does not change the effect that an increasing $\tau$ has on $\max_k(\Re(\lambda_k))$, and that increasing the delay $\tau$ continues to act as a stabilising agent for Turing instabilities, with a shifting of the spatially homogeneous inner arc.

\begin{figure}[H]
    \centering
    \begin{subfigure}[b]{0.45\textwidth}
        \centering
        \includegraphics[width=7cm,height=5cm]{fixbif21.png}
        \caption{$\tau=0$.}
        \label{}
    \end{subfigure}
    \hfill
    \begin{subfigure}[b]{0.45\textwidth}
        \centering
        \includegraphics[width=7cm,height=5cm]{fixbif22.png}
        \caption{$\tau=0.5$}
        \label{}
    \end{subfigure}
    \hfill
    \begin{subfigure}[b]{0.45\textwidth}
        \centering
        \includegraphics[width=7cm,height=5cm]{fixbif23.png}
        \caption{$\tau=1$}
        \label{}
    \end{subfigure}
    \hfill
    \begin{subfigure}[b]{0.45\textwidth}
        \centering
        \includegraphics[width=7cm,height=5cm]{fixbif24.png}
        \caption{$\tau=1.5$.}
        \label{}
    \end{subfigure}
    \caption{$\max_k(\Re(\lambda_k))$ computed over $(a,b)$ parameter space by solving \eqref{realfixbif} and \eqref{complexfixbif}, with $\epsilon^2=0.1$, $L^2=9/2$. As $\tau$ increases, $|\max_k(\Re(\lambda_k))|$ decreases. Contour lines for $\Re(\lambda_0)=0$ and $\max_k(\Re(\lambda_k))=0$ overlayed, indicated Turing instability region. $\max_k$ taken over $k\in[0, 50]$ at discrete intervals of $1$.}
    \label{fig:fixbif2}
\end{figure}
\section{Investigation of Variation in Initial and Boundary Conditions}
In this section, the robustness of the results obtained in \cite{gaffmonk} are examined under varying of initial conditions and boundary conditions. We first consider the sensitivity of pattern formation in the context of fixed time delay to varying initial conditions. Three different sets of initial conditions are considered. $\text{IC}_1$ corresponds to the initial conditions used in \cite{gaffmonk}. The functional form of $\text{IC}_1$ can be found in appendix \ref{section:appA}. The following are the forms of the remaining initial conditions used

\begin{align}\label{ICs}
\text{IC}_2&:\quad\quad\quad\begin{pmatrix}u_0\\v_0\end{pmatrix}=\begin{pmatrix}u_\star(1+r)\\v_\star(1+r)\end{pmatrix}\quad r\sim\mathcal{N}\left(0,0.01^2\right)\\
\text{IC}_3&:\quad\quad\quad\begin{pmatrix}u_0\\v_0\end{pmatrix}=\begin{pmatrix}u_\star(1+r)\\v_\star(1+r)\end{pmatrix}\quad r\sim\mathcal{N}\left(0,0.1^2\right).
\end{align}
We note that computationally a fixed random seed was set. The model parameters used match those used in \cite{gaffmonk}, with $(a,b)=(0.1,0.9)$, and a constant history function equal to the initial conditions was used. The results in figures \ref{fig:figtau0}, \ref{fig:figtau1}, \ref{fig:figtau2}, \ref{fig:figtau4}, and \ref{fig:figtau8} show the pattern formation observed for each of the initial conditions for varying fixed time delay $\tau\in\{0,1,2,4,8 \}$.

\begin{figure}[H]
    \centering
    \begin{subfigure}[b]{0.32\textwidth}
        \centering
        \includegraphics[width=5cm,height=4.5cm]{gaff0.png}
        \caption{$\text{IC}_1$}
        \label{}
    \end{subfigure}
    \hfill
    \begin{subfigure}[b]{0.32\textwidth}
        \centering
        \includegraphics[width=5cm,height=4.5cm]{ic20.png}
        \caption{$\text{IC}_2$}
        \label{}
    \end{subfigure}
    \hfill
    \begin{subfigure}[b]{0.32\textwidth}
        \centering
        \includegraphics[width=5cm,height=4.5cm]{ic30.png}
        \caption{$\text{IC}_3$}
        \label{}
    \end{subfigure}
    \caption{Numerical simulations of \eqref{fixed2} showing comparison of varying ICs for $\tau=0$. Boundary conditions given by \eqref{neumannbc}. $(a,b)=(0.1,0.9)$, $\epsilon^2=0.001$, $L^2=9/2$. }
    \label{fig:figtau0}
\end{figure}
\begin{figure}[H]
    \centering
    \begin{subfigure}[b]{0.32\textwidth}
        \centering
        \includegraphics[width=5cm,height=4.5cm]{gaff1.png}
        \caption{$\text{IC}_1$}
        \label{}
    \end{subfigure}
    \hfill
    \begin{subfigure}[b]{0.32\textwidth}
        \centering
        \includegraphics[width=5cm,height=4.5cm]{ic21.png}
        \caption{$\text{IC}_2$}
        \label{}
    \end{subfigure}
    \hfill
    \begin{subfigure}[b]{0.32\textwidth}
        \centering
        \includegraphics[width=5cm,height=4.5cm]{ic31.png}
        \caption{$\text{IC}_3$}
        \label{}
    \end{subfigure}
    \caption{Numerical simulations of \eqref{fixed2} showing comparison of varying ICs for $\tau=1$. Boundary conditions given by \eqref{neumannbc}. $(a,b)=(0.1,0.9)$, $\epsilon^2=0.001$, $L^2=9/2$.}
    \label{fig:figtau1}
\end{figure}
\begin{figure}[H]
    \centering
    \begin{subfigure}[b]{0.32\textwidth}
        \centering
        \includegraphics[width=5cm,height=4.5cm]{gaff2.png}
        \caption{$\text{IC}_1$}
        \label{}
    \end{subfigure}
    \hfill
    \begin{subfigure}[b]{0.32\textwidth}
        \centering
        \includegraphics[width=5cm,height=4.5cm]{ic22.png}
        \caption{$\text{IC}_2$}
        \label{}
    \end{subfigure}
    \hfill
    \begin{subfigure}[b]{0.32\textwidth}
        \centering
        \includegraphics[width=5cm,height=4.5cm]{ic32.png}
        \caption{$\text{IC}_3$}
        \label{}
    \end{subfigure}
    \caption{Numerical simulations of \eqref{fixed2} showing comparison of varying ICs for $\tau=2$.Boundary conditions given by \eqref{neumannbc}. $(a,b)=(0.1,0.9)$, $\epsilon^2=0.001$, $L^2=9/2$.}
    \label{fig:figtau2}
\end{figure}
\begin{figure}[H]
    \centering
    \begin{subfigure}[b]{0.32\textwidth}
        \centering
        \includegraphics[width=5cm,height=4.5cm]{gaff4.png}
        \caption{$\text{IC}_1$}
        \label{}
    \end{subfigure}
    \hfill
    \begin{subfigure}[b]{0.32\textwidth}
        \centering
        \includegraphics[width=5cm,height=4.5cm]{ic24.png}
        \caption{$\text{IC}_2$}
        \label{}
    \end{subfigure}
    \hfill
    \begin{subfigure}[b]{0.32\textwidth}
        \centering
        \includegraphics[width=5cm,height=4.5cm]{ic34.png}
        \caption{$\text{IC}_3$}
        \label{}
    \end{subfigure}
    \caption{Numerical simulations of \eqref{fixed2} showing comparison of varying ICs for $\tau=4$. Boundary conditions given by \eqref{neumannbc}. $(a,b)=(0.1,0.9)$, $\epsilon^2=0.001$, $L^2=9/2$.}
    \label{fig:figtau4}
\end{figure}
\begin{figure}[H]
    \centering
    \begin{subfigure}[b]{0.32\textwidth}
        \centering
        \includegraphics[width=5cm,height=4.5cm]{gaff8.png}
        \caption{$\text{IC}_1$}
        \label{}
    \end{subfigure}
    \hfill
    \begin{subfigure}[b]{0.32\textwidth}
        \centering
        \includegraphics[width=5cm,height=4.5cm]{ic28.png}
        \caption{$\text{IC}_2$}
        \label{}
    \end{subfigure}
    \hfill
    \begin{subfigure}[b]{0.32\textwidth}
        \centering
        \includegraphics[width=5cm,height=4.5cm]{ic38.png}
        \caption{$\text{IC}_3$}
        \label{}
    \end{subfigure}
    \caption{Numerical simulations of \eqref{fixed2} showing comparison of varying ICs for $\tau=8$. Boundary conditions given by \eqref{neumannbc}. $(a,b)=(0.1,0.9)$, $\epsilon^2=0.001$, $L^2=9/2$.}
    \label{fig:figtau8}
\end{figure}

It can be seen that the final pattern is sensitive to the choice of initial conditions, and that, intuitively, the larger $\sigma_{\text{IC}}$ used in $\text{IC}_3$, compared to that of $\text{IC}_2$ results in a faster onset of pattern formation. We see from considering the timescales as to which pattern formation occurs however, that although the time taken until onset of patterning varies with different initial conditions, the increase in time-to-pattern with an increasing time delay is consistent independent of the initial conditions chosen. By considering the varying $x$-axis, we also note that in each case, this relationship seems to be linear. We formalise this in section \ref{section:delaypatt}.

Numerical results were also simulated to study the effects of a temporal variation in the history function. A history function was set as $h(t)=u_\star(1+r\sin(\omega t))$ for $t\in[-\tau,0)$, where $r$ is the random variable used in $\text{IC}_2$. Simulations were conducted for varying $\tau$ and $\omega$. Preliminary simulations, which can be found in appendix \ref{section:appB}, show that this type of variation in history does not have a significant effect on the results seen.

Finally, we consider the effect of varying boundary conditions. Motivated from the analysis in \cite{krausemixed}, homogeneous Dirichlet boundary conditions are implemented for the activator term, and homogeneous Neumann boundary conditions implemented for the inhibitor term. Thus, we have that, on a domain $\Omega=[0,1]$, $u=0$ and $\frac{\partial v}{\partial t}=0$ at $x=0, 1$. These conditions are implemented numerically following the methodology outlined in section \ref{section:numimp}. The results in figures \ref{fig:bctau1}, \ref{fig:bctau2}, and \ref{fig:bctau3} were generated using $\text{IC}_2$, with a varying $\tau\in\{0,1,8\}$, and show the comparison between numerical simulations generated with homogeneous Neumann conditions for both $u$ and $v$, indicated as $\text{BC}_1$, and those generated with homogeneous Dirichlet conditions for $u$, indicated as $\text{BC}_2$.

\begin{figure}[H]
    \centering
    \begin{subfigure}[b]{0.45\textwidth}
        \centering
        \includegraphics[width=7cm,height=5.5cm]{ic20.png}
        \caption{$\text{BC}_1$}
        \label{}
    \end{subfigure}
    \hfill
    \begin{subfigure}[b]{0.45\textwidth}
        \centering
        \includegraphics[width=7cm,height=5.5cm]{bc0.png}
        \caption{$\text{BC}_2$}
        \label{}
    \end{subfigure}
    \caption{Comparison of varying BCs for $\tau=0$. Generated with $\text{IC}_2$. $(a,b)=(0.1,0.9)$, $\epsilon^2=0.001$, $L^2=9/2$.}
    \label{fig:bctau1}
\end{figure}

\begin{figure}[H]
    \centering
    \begin{subfigure}[b]{0.45\textwidth}
        \centering
        \includegraphics[width=7cm,height=5.5cm]{ic21.png}
        \caption{$\text{BC}_1$}
        \label{}
    \end{subfigure}
    \hfill
    \begin{subfigure}[b]{0.45\textwidth}
        \centering
        \includegraphics[width=7cm,height=5.5cm]{bc1.png}
        \caption{$\text{BC}_2$}
        \label{}
    \end{subfigure}
    \caption{Comparison of varying BCs for $\tau=1$. Generated with $\text{IC}_2$. $(a,b)=(0.1,0.9)$, $\epsilon^2=0.001$, $L^2=9/2$.}
    \label{fig:bctau2}
\end{figure}

\begin{figure}[H]
    \centering
    \begin{subfigure}[b]{0.45\textwidth}
        \centering
        \includegraphics[width=7cm,height=5.5cm]{ic28.png}
        \caption{$\text{BC}_1$}
        \label{}
    \end{subfigure}
    \hfill
    \begin{subfigure}[b]{0.45\textwidth}
        \centering
        \includegraphics[width=7cm,height=5.5cm]{bc8.png}
        \caption{$\text{BC}_2$}
        \label{}
    \end{subfigure}
    \caption{Comparison of varying BCs for $\tau=8$. Generated with $\text{IC}_2$. $(a,b)=(0.1,0.9)$, $\epsilon^2=0.001$, $L^2=9/2$.}
    \label{fig:bctau3}
\end{figure}

We note that, although changing the boundary conditions for the activator term $u$ to homogeneous Dirichlet conditions affects the type of patterns we may see (number and amplitude of spikes), this change does not affect the increased timescales, caused by an increase in time delay, on which onset of patterning occurs.

\section{Relationship Between Time-To-Pattern and time delay}\label{section:delaypatt}

We aim to show that, for small $\tau$ and small $L$, the linear theory provides a good approximation to the time taken until pattern formation occurs, and in fact, the relationship between $\tau$ and time-to-pattern under these conditions is linear. We also show that, through full numerical solutions, the relationship between $\tau$ and time-to-pattern on a longer time-scale for larger $\tau$ is also linear. We first consider the former.

To minimise the effect of nonlinearity in the dynamics, we restrict the domain size to $L^2=1/5$. Shrinking the domain results in fewer unstable modes and thus less competition for the dominant mode, resulting in a better approximation of the linear theory. This finite size effect can be seen in figure \ref{fig:compardisp}, where $\Re(\lambda_k)$ is plotted against $k$ for two different domain sizes, for a given $(a,b,\tau)$. Due to numerical restrictions when using Chebfun in finding roots of the characteristic equation \eqref{characfix}, only $\tau\leq1.6$ is considered. Similar to the initial conditions used for previous numerical simulations (figure \ref{fig:fixedsim2}), we take a small perturbation in the activator term, $\hat{u}(t)$, such that $\hat{u}(0)=ru_\star$, where $r$ is a small Gaussian random variable, $r\sim\mathcal{N}\left(0,\sigma_{\text{IC}}^2\right)$,
for some standard deviation of the initial perturbation $\sigma_{\text{IC}}$.

The linear theory suggests that at some time $t=T$, the perturbation will be of the form $\hat{u}(T)\sim A_k(T)\cos(k\pi x)$, where $k$ is the dominant mode and $A_k(T)$ denotes the corresponding Fourier coefficient at time $t=T$. For a given parameter set $(a,b,\epsilon^2,\tau,L)$, we can solve the characteristic equation \eqref{characfix}, and plot $\Re(\lambda_k)$
against $k$, to determine the dominating mode $k$ and the corresponding eigenvalue, or growth rate, $\lambda_k$. We then use this information in the following manner: A Fast Fourier Transform to decompose the initial conditions into a Fourier series is used, and the coefficient $A_k(0)$ for the dominating $k$ is computed. When the perturbation $\hat{u}$ has grown sufficiently, in absolute value, beyond a threshold where pattern formation is considered, we call this time $t=T$, and again determine the Fourier coefficient $A_k(T)$ of the fastest-growing mode $k$. More specifically, the time $T$ is the first such that $\max_x|u(T,x)-u_\star|>threshold$, namely the first time such that any solution point across the whole spatial domain is large enough, in absolute difference, from the steady-state. Finally, using the relation $A_k(T)\sim A_k(0)e^{\lambda_k T}$, we rearrange for $T$ and thus compute a linear approximation for time-to-pattern as
\begin{equation}\label{ttprelation}
    T=\frac{1}{\lambda_k}\ln\left(\frac{A_k(T)}{A_k(0)}\right).
\end{equation}
We consider an example case for $(a,b,\tau)=(0.4,1.8,0.2)$. The standard deviation for the random variable $r$ is chosen as $\sigma_{\text{IC}}=10^{-5}$, and the threshold value at $0.1$. A very small perturbation was used as a means to improve the accuracy of the linear theory.
\begin{figure}[H]
    \centering
    \begin{subfigure}[b]{0.45\textwidth}
        \centering
        \includegraphics[width=7cm,height=5.5cm]{compdisp1.png}
        \caption{Dispersion curve plotted with domain size $L^2=1/5$. Curve produced by varying $k\in[0,5]$ at regular intervals of $0.1$. Discrete values of $k$ overlayed as scatter points. }
        \label{fig:compdisp1}
    \end{subfigure}
    \hfill
    \begin{subfigure}[b]{0.45\textwidth}
        \centering
        \includegraphics[width=7cm,height=5.5cm]{compdisp2.png}
        \caption{Dispersion curve plotted with domain size $L^2=9/2$. Curve produced by varying $k\in[0,60]$ at regular intervals of $1$.}
        \label{fig:compdisp2}
    \end{subfigure}
    \caption{Dispersion curves of the characterstic equation given in \eqref{characfic} plotted for $(a,b,\tau)=(0.4,1.8,0.2)$ and $\epsilon^2=0.001$. A larger $L$ results in more unstable modes $\lambda_k$ such that $\Re(\lambda_k)>0$. }
    \label{fig:compardisp}
\end{figure}
Using $\epsilon^2=0.001$, on the domain size $L=2\sqrt{0.05}$, figure \ref{fig:compdisp1} suggests that, from the linear theory, the dominant mode is $k=1$ with dominant eigenvalue $\lambda_1=0.2356$. Since $k=1$ is the dominant mode, we compute the first Fourier coefficient of the initial conditions, $A_1(0)$, as $A_1(0)=7.95(3.s.f)\times10^{-8}$. To find $A_1(T)$, a numerical simulation is run until the solution of the activator $u$ has grown, in absolute value, to a threshold value of $0.1$. Figure \ref{fig:Tfc} shows the numerical solution $u(T)$ at the point where this threshold value has been met, as well as a scatter plot of the Fourier coefficients $A_k(T)$, $k\neq0$.
\begin{figure}[H]
    \centering
    \begin{subfigure}[b]{0.45\textwidth}
        \centering
        \includegraphics[width=7cm,height=6cm]{Tu.png}
        \caption{Numerical solution $u(T)$ at $t=T$}
        \label{uT}
    \end{subfigure}
    \hfill
    \begin{subfigure}[b]{0.45\textwidth}
        \centering
        \includegraphics[width=7cm,height=6cm]{FCs.png}
        \caption{Absolute Fourier coefficients of $u(T)$, for $k\in[1,50]$.}
        \label{fig:uTfc}
    \end{subfigure}
    \caption{Numerical solution of \eqref{fixed2} at $t=T$ with homogeneous Neumann boundary conditions implemented. Initial perturbation from steady-state with $\sigma_{\text{IC}}=10^{-5}$. First $50$ Fourier coefficients for $u(T)$ plotted, with $(a,b)=(0.4,1.8)$, time delay $\tau=0.2$ and $\epsilon^2=0.001$, $L=2\sqrt{0.05}$.}
    \label{fig:Tfc}
\end{figure}
As seen in figure \ref{fig:uTfc}, the Fourier coefficient corresponding to $k=1$ is given as $0.0262(3.s.f)$. The approximated time-to-pattern, as predicted by linear theory, for $(a,b,\tau)=(0.4,1.8,0.2)$ and the given initial conditions is thus computed as
\begin{equation}
    T=\frac{1}{\lambda_1}\ln\left(\frac{A_1(T)}{A_1(0)}\right)=\frac{1}{0.2356}\ln\left(\frac{0.0262}{7.95\times10^{-8}}\right)=53.8(3.s.f).
\end{equation}
It was found through numerical solutions that the `true' time-to-pattern is $\approx57.5(3sf)$.
We use `true' time-to-pattern here to mean the time taken for a perturbation to grow above a threshold value found through full numerical solutions. This process can be repeated for varying $(a,b,\tau)$, and figures \ref{fig:ttp1}, \ref{fig:ttp2}, \ref{fig:ttp3} show the predicted time-to-pattern plotted against $\tau$ and compared with the `true' time-to-pattern for three different parameter sets. The time delay is varied here over $\tau\in[0,1.6]$ at intervals of $0.2$.

\begin{figure}[H]
    \centering
    \begin{subfigure}[b]{0.32\textwidth}
        \centering
        \includegraphics[width=5cm,height=5cm]{ttp1.png}
        \caption{$(a,b)=(0.4,1.8)$.}
        \label{fig:ttp1}
    \end{subfigure}
    \hfill
    \begin{subfigure}[b]{0.32\textwidth}
        \centering
        \includegraphics[width=5cm,height=5cm]{ttp2.png}
        \caption{$(a,b)=(0.1,0.9)$.}
        \label{fig:ttp2}
    \end{subfigure}
    \hfill
    \begin{subfigure}[b]{0.32\textwidth}
        \centering
        \includegraphics[width=5cm,height=5cm]{ttp3.png}
        \caption{$(a,b)=(0.2,1.3)$.}
        \label{fig:ttp3}
    \end{subfigure}
    \caption{Predicted vs `true' time-to-pattern for numerical solution of \eqref{fixed2} with homogeneous Neumann boundary conditions implemented. Initial perturbation from steady-state with $\sigma_{\text{IC}}= 10^{-5}$. Predicted time-to-pattern computed using the relationship \eqref{ttprelation}, for three different parameter sets, with $L=2\sqrt{0.05}$, $\epsilon^2=0.001$, and $\tau\in[0,1.6]$.}
    \label{}
\end{figure}

Now, through full numerical solutions, we show a linear relationship between $\tau$ and time-to-pattern on a longer time-scale. Varying $\tau\in[1,16]$ at regular intervals of $1$, for two different parameter sets $(a,b)=\{(0.1,0.9),(0.4,0.8)\}$, we compute the time taken for a perturbation to grow up to a threshold value, and plot the results. Figure \ref{fig:linperturb1a} shows the results for $(a,b)=(0.1,0.9)$ with an initial perturbation of standard-deviation $\sigma_{\text{IC}}=10^{-5}$ and threshold value $0.1$. Figure \ref{fig:linperturb1b} shows the results for the same parameter values but with an initial perturbation of standard-deviation $\sigma_{\text{IC}}=0.01$ and threshold value $2$. Figures \ref{fig:linperturb2a} and \ref{fig:linperturb2b} show analogous results but for $(a,b)=(0.4,0.8)$. The simulations were run with $\epsilon^2=0.001$ and $L^2=9/2$.
\begin{figure}[H]
    \centering
    \begin{subfigure}[b]{0.45\textwidth}
        \centering
        \includegraphics[width=7cm,height=6cm]{longlin2.png}
        \caption{$\sigma_{\text{IC}}=10^{-5}$ and threshold $0.1$.}
        \label{fig:linperturb1a}
    \end{subfigure}
    \hfill
    \begin{subfigure}[b]{0.45\textwidth}
        \centering
        \includegraphics[width=7cm,height=6cm]{longlin1.png}
        \caption{$\sigma_{\text{IC}}=0.01$ and threshold $2$}
        \label{fig:linperturb1b}
    \end{subfigure}
    \caption{Time-to-pattern for full numerical solutions of \eqref{fixed2} plotted against $\tau\in[1,16]$ for two different initial perturbations and threshold values. $(a,b)=(0.1,0.9)$, $\epsilon^2=0.001$ and domain size $L^2=9/2$.}
    \label{fig:linperturb1}
\end{figure}

\begin{figure}[H]
    \centering
    \begin{subfigure}[b]{0.45\textwidth}
        \centering
        \includegraphics[width=7cm,height=6cm]{longlin3.png}
        \caption{$\sigma_{\text{IC}}=10^{-5}$ and threshold $0.1$}
        \label{fig:linperturb2a}
    \end{subfigure}
    \hfill
    \begin{subfigure}[b]{0.45\textwidth}
        \centering
        \includegraphics[width=7cm,height=6cm]{longlin4.png}
        \caption{$\sigma_{\text{IC}}=0.01$ and threshold $2$}
        \label{fig:linperturb2b}
    \end{subfigure}
    \caption{Time-to-pattern for full numerical solutions of \eqref{fixed2} plotted against $\tau\in[1,16]$ for two different initial perturbations and threshold values. $(a,b)=(0.4,0.8)$, $\epsilon^2=0.001$ and domain size $L^2=9/2$.}
    \label{fig:linperturb2}
\end{figure}


\section{Summary}

In this chapter, we presented the biological motivation for studying the LI variant of the Schnakenberg model with incorporated fixed delay.

Linear theory suggested that for the LI model, time delay can act as a stabilising agent for Turing instabilities, increasing the parameter region where Turing instabilities can occur. Through both linear analysis on a small scale, and full numerical solutions on a larger scale, a linear relationship between time delay and time-to-pattern was presented, and an increasing delay was shown to, depending on the parameter chosen, increase the time taken for pattern formation to occur, or increase the time taken for perturbations to fully decay.

Numerical results in this chapter were also systematically tested with a varying of initial and boundary conditions, and a temporal variation in the history function. Simulations suggested that the increase in time-to-pattern with an increase in delay is robust to these variations. We therefore look for ways to remedy the problems caused by a fixed delay. Considering the complexity and stochastic nature of pattern formation on a cellular level leads us to consider modelling the time delay as a distrbution, which we consider in the next chapter.

\begin{figure}[H]
    \centering
    \begin{subfigure}[b]{0.45\textwidth}
        \centering
        \includegraphics[width=7cm,height = 5.5cm]{nopatt1.png}
        \caption{$\tau=0$. No pattern formation after $t=10^4$. }
        \label{}
    \end{subfigure}
    \hfill
    \begin{subfigure}[b]{0.45\textwidth}
        \centering
        \includegraphics[width=7cm,height = 5.5cm]{nopatt2.png}
        \caption{$\tau=1.$ No pattern formation after $t=10^4$.}
        \label{}
    \end{subfigure}
    \caption{Numerical simulations of \eqref{fixed2} showing no pattern formation with $(a,b)=(0.4,0.4)$, $\epsilon^2=0.001$ and $L^2=9/2$. Homogeneous Neumann boundary conditions implemented for both $u$ and $v$. Initial conditions given as a random perturbation from homogeneous steady-state, as described in \eqref{firstic}}
    \label{fig:fixedsim1}
\end{figure}1

\begin{figure}[H]
    \centering
    \begin{subfigure}[b]{0.45\textwidth}
        \centering
        \includegraphics[width=7cm,height = 5.5cm]{patt1.png}
        \caption{$\tau=0$. Distinct spikes formed at $t\approx7$ }
        \label{}
    \end{subfigure}
    \hfill
    \begin{subfigure}[b]{0.45\textwidth}
        \centering
        \includegraphics[width=7cm,height = 5.5cm]{patt2.png}
        \caption{$\tau=1$. Distinct spikes formed at $t\approx50$.}
        \label{}
    \end{subfigure}
    \caption{Numerical simulations of \eqref{fixed2} showing pattern formation for $(a,b)=(0.1,0.9)$, $\epsilon^2=0.001$ and $L^2=9/2$. Homogeneous Neumann boundary conditions implemented for both $u$ and $v$. Initial conditions given as a random perturbation from homogeneous steady-state, as described in \eqref{firstic}}
    \label{fig:fixedsim2}
\end{figure}
These figures were produced using an initial condition $(u_0,v_0)$ given as a random Gaussian perturbation from the homogeneous steady-state. Namely,
\begin{equation}\label{firstic}
\begin{pmatrix}u_0\\v_0\end{pmatrix}=\begin{pmatrix}u_\star(1+r)\\v_\star(1+r)\end{pmatrix},
\end{equation}
where $r$ is a random variable such that $r\sim\mathcal{N}\left(0,0.01^2\right)$. The notation $r\sim\mathcal{N}\left(\mu,\sigma_{\text{IC}}^2\right)$ denotes a Normally distributed random variable $r$ with mean $\mu$ and standard deviation of the initial perturbation $\sigma_{\text{IC}}$. A constant history function equal to the initial conditions was used.
\subsection{Bifurcation Analysis}\label{section:fixedbif}
The Turing plot produced in figure \ref{fig:turingspace}, computed using the conditions in \eqref{cond1} and \eqref{cond2}, is a bifurcation diagram indicating regions of Turing instability. We note two separate curves which separate the parameter space into its distinct regions. These will be referred to as the `lines of stability'.
These two curves are indicated in figure \ref{fig:bif0}. The inner arc corresponds to the $(a,b)$ such that $\Re(\lambda_k)=0$ for the
spatially homogeneous characteristic equation, $\mathcal{D}_k=0$ when $k=0$. For $\tau=0$, this corresponds exactly to equating conditions $\eqref{cond1}$ to 0. The outer boundary is the points $(a,b)$ such that $\max_k\Re(\lambda_k)=0$ for the spatially inhomogeneous characteristic equation, $\mathcal{D}_k$ when $k\neq0$. For $\tau=0$, this is identical to equating to the conditions $\eqref{cond2}$ to 0. By letting $\lambda_k=x_k+iy_k$ for $x,y\in\mathbb{R}$, we split the characteristic equation $\mathcal{D}_k=0$ into its real and imaginary parts, $\mathcal{D}_k^{\Re}=0$ and $\mathcal{D}_k^{\Im}=0$, given by
\begin{align}\label{realfix}
\mathcal{D}_k^{\Re}&=x_k^2-y_k^2+\alpha_kx_k+\beta_k+e^{-x_k\tau}[\gamma_kx_k\cos(-y_k\tau)-\gamma_ky_k\sin(-y_k\tau)+\delta_k\cos(-y_k\tau)]=0,\\
\mathcal{D}_k^{\Im}&=2x_ky_k+\alpha_ky_k+e^{-x_k\tau}[\gamma_kx_k\sin(-y_k\tau)+\gamma_ky_k\cos(-y_k\tau)+\delta_k\sin(-y_k\tau)]=0.\label{complexfix}
\end{align}
By setting $\Re(\lambda_k)=x_k=0$ in equations \eqref{realfix} and \eqref{complexfix}, the real and imaginary parts of $\mathcal{D}_k$ can be simplified to
\begin{align}\label{realfixbif}
  \mathcal{D}_k^{\Re}&=-y_k^2\beta_k+[-\gamma_ky_k\sin(-y_k\tau)+\delta_k\cos(-y_k\tau)],\\
  \mathcal{D}_k^{\Im}&=\alpha_ky_k+[\gamma_ky_k\cos(-y_k\tau)+\delta_k\sin(-y_k\tau)].\label{complexfixbif}
\end{align}
For a fixed $\tau$ and $b$, the roots of \eqref{realfixbif} and \eqref{complexfixbif} (at $k=0$) can be found for $a$ and $\Im(\lambda_k)$.
Taking the $\max_k(a)$, a curve can be plotted in the $(a,b)$ parameter space for the outer boundary (and the inner arc) resulting in a bifurcation diagram of distinct regions where Turing instabilities can occur. We use a relatively large $L^2=9/2$ and so the bifurcation diagram computed in this manner for $\tau=0$ should be a good approximation to the Turing space plot produced in figure \ref{fig:turingspace}. Figure \ref{fig:bif0} shows the bifurcation plot produced in this manner for $\tau=0$ alongside the Turing space plot in figure \ref{fig:turingspace} for comparison.

\begin{figure}[H]
    \centering
    \begin{subfigure}[b]{0.47\textwidth}
        \centering
        \includegraphics[width=7cm,height = 5.5cm]{bif0.png}
        \caption{Stability lines for $\tau=0$ computed by solving characteristic equation with $\epsilon^2=0.001$, $L^2=9/2$.}
        \label{fig:bif0}
    \end{subfigure}
    \hfill
    \begin{subfigure}[b]{0.47\textwidth}
        \centering
        \includegraphics[width=7cm,height = 5.5cm]{turingspace.png}
        \caption{Turing space as in figure \ref{fig:turingspace} plotted from parameters $(a,b)$ satisfying conditions Turing conditions.}
        \label{}
    \end{subfigure}
    \caption{Comparison of Turing instability region for $\tau=0$ computed via characterstic equation(s) (\eqref{realfixbif} and \eqref{complexfixbif}) against Turing instability region computed via Turing conditions (\eqref{cond1} and \eqref{cond2}). Parameter space chosen as $(a,b)\in[0,1.4]\times[0,2]$.}
    \label{fig:tspace1}
\end{figure}

Figure \ref{fig:tspacetau} shows the stability curves computed for a varying $\tau\in\{0,0.5,1,1.5\}$. It can be seen that, whilst the outer boundary computed from the characteristic equation for the spatially inhomogeneous model stays the same, the inner arc computed from using the characteristic equation from the spatially homogeneous model shifts to the left, increasing the region of parameter space for which Turing instabilities can occur. This observation matches the results produced in \cite{william} for the spatially homogeneous LI model. This is also a similar result to that observed in \cite{fadai2} for the Gierer-Meinhardt model. We see that for the LI model where delay-terms are placed solely in the activator dynamics, time delay acts as a stabilising agent for pattern formation and increases the parameter space where Turing instabilities can occur.
\begin{figure}[H]
        \centering
        \includegraphics[width=12cm,height = 9cm]{tspacetau.png}
        \caption{Stability lines for $\tau\in\{0,0.5,1,1.5\}$ computed by solving \eqref{realfixbif} and \eqref{complexfixbif}. $\epsilon^2=0.001$, $L^2=9/2$.}
        \label{fig:tspacetau}
\end{figure}
We verify the results of figure \ref{fig:tspacetau} through numerical simulations. Three parameter points, $(a,b)=\{(0.12,0.5),(1.2,1.75),(1.2,1.85)\}$ are indicated in figure \ref{fig:tspacetau}. At $(a,b,\tau)=(0.12,0.5,0)$, linear theory suggests that there will be no pattern formation, but at $(a,b,\tau)=(0.12,0.5,1.5)$ there will be a Turing instability and thus patterns will form.
The parameter region in the bottom left of the parameter space is a delicate region that can exhibit both Turing and Hopf bifurcations, leading to complex spatio-temporal behaviours. This type of dynamics in reaction-diffusion systems has been studied more extensively in \cite{krausefixed,jiang}. Although the linear theory is unable to provide information about the more intricate nonlinear dynamics, it can predict the expected type of behaviour for certain parameter values. To show a change in behaviour as $\tau$ changes from $0$ to $1.5$, from a temporally oscillating solution, to one exhibiting a Turing pattern, we increase the diffusive ratio to $\epsilon^2=0.1$. This result can be seen in figure \ref{fig:testturing}. The linear theory also suggests that for all $\tau\in\{0,0.5,1,1.5\}$, pattern formation will occur for $(a,b)=(1.2,1.85)$, but not for $(a,b)=(1.2,1.75)$. Figures \ref{fig:testturing2} and \ref{fig:testturing3} show the results for numerical simulations at $(a,b)=\{(1.2,1.75),(1.2,1.85)\}$ for $\tau=\{0,0.5,1,1.5\}$.
\begin{figure}[h]
    \centering
    \begin{subfigure}[b]{0.45\textwidth}
        \centering
        \includegraphics[width=7cm,height=5cm]{toscill.png}
        \caption{$\tau=0$. Oscillations seen.}
        \label{}
    \end{subfigure}
    \hfill
    \begin{subfigure}[b]{0.45\textwidth}
        \centering
        \includegraphics[width=7cm,height=5cm]{tpattpred.png}
        \caption{$\tau=1.5$. Pattern formation seen.}
        \label{}
    \end{subfigure}
    \caption{Numerical simulations of \eqref{fixed2} produced with parameters $(a,b)=(0.12,0.5)$, for $\tau=0,1.5$. $\epsilon^2=0.1$ and $L^2=9/2$. Homogeneous Neumann boundary conditions implemented for both $u$ and $v$. Initial conditions given as a random perturbation from homogeneous steady-state, as described in \eqref{firstic}. Linear theory in figure \ref{fig:tspacetau} suggests we see pattern formation at $\tau=1.5$ but not at $\tau=0$.}
    \label{fig:testturing}
\end{figure}

\begin{figure}[H]
    \centering
    \begin{subfigure}[b]{0.45\textwidth}
        \centering
        \includegraphics[width=7cm,height=4cm]{p3t0.png}
        \caption{$\tau=0$.}
        \label{}
    \end{subfigure}
    \hfill
    \begin{subfigure}[b]{0.45\textwidth}
        \centering
        \includegraphics[width=7cm,height=4cm]{p3t05.png}
        \caption{$\tau=0.5$}
        \label{}
    \end{subfigure}
    \hfill
    \begin{subfigure}[b]{0.45\textwidth}
        \centering
        \includegraphics[width=7cm,height=4cm]{p3t1.png}
        \caption{$\tau=1$}
        \label{}
    \end{subfigure}
    \hfill
    \begin{subfigure}[b]{0.45\textwidth}
        \centering
        \includegraphics[width=7cm,height=4cm]{p3t15.png}
        \caption{$\tau=1.5$.}
        \label{}
    \end{subfigure}
    \caption{Numerical simulations of \eqref{fixed2} for $(a,b)=(1.2,1.75)$. $\epsilon^2=0.001$ and $L^2=9/2$. Homogeneous Neumann boundary conditions implemented for both $u$ and $v$. Initial conditions given as a random perturbation from homogeneous steady-state, as described in \eqref{firstic}. We see no pattern formation for $\tau\in\{0,0.5,1,1.5\}$ as suggested by linear theory, seen in figure \ref{fig:tspacetau}.}
    \label{fig:testturing2}
\end{figure}

\begin{figure}[H]
    \centering
    \begin{subfigure}[b]{0.45\textwidth}
        \centering
        \includegraphics[width=7cm,height=4cm]{p2t0.png}
        \caption{$\tau=0$.}
        \label{}
    \end{subfigure}
    \hfill
    \begin{subfigure}[b]{0.45\textwidth}
        \centering
        \includegraphics[width=7cm,height=4cm]{p2t05.png}
        \caption{$\tau=0.5$}
        \label{}
    \end{subfigure}
    \hfill
    \begin{subfigure}[b]{0.45\textwidth}
        \centering
        \includegraphics[width=7cm,height=4cm]{p2t1.png}
        \caption{$\tau=1$}
        \label{}
    \end{subfigure}
    \hfill
    \begin{subfigure}[b]{0.45\textwidth}
        \centering
        \includegraphics[width=7cm,height=4cm]{p2t15.png}
        \caption{$\tau=1.5$.}
        \label{}
    \end{subfigure}
    \caption{Numerical simulations of \eqref{fixed2} for $(a,b)=(1.2,1.85)$. $\epsilon^2=0.001$ and $L^2=9/2$. Homogeneous Neumann boundary conditions implemented for both $u$ and $v$. Initial conditions given as a random perturbation from homogeneous steady-state, as described in \eqref{firstic}. We see pattern formation on an increasing time-scale for $\tau\in\{0,0.5,1,1.5\}$ as suggested by linear theory, seen in figure \ref{fig:tspacetau}. Increasing time-scale seen from changing $x$-axis limits.}
    \label{fig:testturing3}
\end{figure}

Figure \ref{fig:tspacetau} shows how the time delay affects the region of Turing instability, but it provides no information as to how $\max_k(\Re(\lambda_k))$ varies as $\tau$ increases over the $(a,b)$ parameter space. In figure \ref{fig:lambdavary} we plot a heatmap of $\max_k(\Re(\lambda_k))$ over the $(a,b)$ parameter space for varying $\tau\in\{0,0.5,1,1.5\}$. Overlayed onto these plots are contour lines corresponding to where $\Re(\lambda_0)=0$ and $\max_k(\Re(\lambda_k))=0$,
highlighting the Turing instability region. The $\max_k$ taken over $k\in[0,50]$ at regular discrete intervals of $1$.
\begin{figure}[H]
    \centering
    \begin{subfigure}[b]{0.45\textwidth}
        \centering
        \includegraphics[width=7cm,height=5cm]{tau0bif.png}
        \caption{$\tau=0$.}
        \label{}
    \end{subfigure}
    \hfill
    \begin{subfigure}[b]{0.45\textwidth}
        \centering
        \includegraphics[width=7cm,height=5cm]{tau05bif.png}
        \caption{$\tau=0.5$}
        \label{}
    \end{subfigure}
    \hfill
    \begin{subfigure}[b]{0.45\textwidth}
        \centering
        \includegraphics[width=7cm,height=5cm]{tau1bif.png}
        \caption{$\tau=1$}
        \label{}
    \end{subfigure}
    \hfill
    \begin{subfigure}[b]{0.45\textwidth}
        \centering
        \includegraphics[width=7cm,height=5cm]{tau15bif.png}
        \caption{$\tau=1.5$.}
        \label{}
    \end{subfigure}
    \caption{$\max_k(\Re(\lambda_k))$ computed over $(a,b)$ parameter space by solving \eqref{realfixbif} and \eqref{complexfixbif}, with $\epsilon^2=0.001$, $L^2=9/2$. As $\tau$ increases, $|\max_k(\Re(\lambda_k))|$ decreases. Contour lines for $\Re(\lambda_0)=0$ and $\max_k(\Re(\lambda_k))=0$ overlayed, indicated Turing instability region. $\max_k$ taken over $k\in[0, 50]$ at discrete intervals of $1$.}
    \label{fig:lambdavary}
\end{figure}
As $\tau$ increases, it can be seen that the absolute value $|\max_k(\Re((\lambda_k)))|$ also decreases. This suggests that for $(a,b)$ values within the Turing instability region, pattern formation will take longer to occur. It also suggests however that for $(a,b)$ such that $\max_k(\Re(\lambda_k))<0$, it will take a longer time for the eigenfunctions with modes $k\neq0$ to decay to a spatially homogeneous steady-state. We note this behaviour in figure \ref{fig:testturing2}, where it can be seen, by carefully considering the timescales, that the time taken for the initial perturbation to fully decay back to a spatially homogeneous steady-state increases as $\tau$ increases. Figure \ref{fig:fixbif2} shows analogous bifurcation diagrams as in figure \ref{fig:lambdavary}, but with $\epsilon^2=0.1$. We note that as the ratio of diffusion constants in the reaction-diffusion system, $\epsilon^2$, moves closer to $1$, the region of parameter space exhibiting Turing instability decreases. It can be observed however that altering $\epsilon^2$ does not change the effect that an increasing $\tau$ has on $\max_k(\Re(\lambda_k))$, and that increasing the delay $\tau$ continues to act as a stabilising agent for Turing instabilities, with a shifting of the spatially homogeneous inner arc.

\begin{figure}[H]
    \centering
    \begin{subfigure}[b]{0.45\textwidth}
        \centering
        \includegraphics[width=7cm,height=5cm]{fixbif21.png}
        \caption{$\tau=0$.}
        \label{}
    \end{subfigure}
    \hfill
    \begin{subfigure}[b]{0.45\textwidth}
        \centering
        \includegraphics[width=7cm,height=5cm]{fixbif22.png}
        \caption{$\tau=0.5$}
        \label{}
    \end{subfigure}
    \hfill
    \begin{subfigure}[b]{0.45\textwidth}
        \centering
        \includegraphics[width=7cm,height=5cm]{fixbif23.png}
        \caption{$\tau=1$}
        \label{}
    \end{subfigure}
    \hfill
    \begin{subfigure}[b]{0.45\textwidth}
        \centering
        \includegraphics[width=7cm,height=5cm]{fixbif24.png}
        \caption{$\tau=1.5$.}
        \label{}
    \end{subfigure}
    \caption{$\max_k(\Re(\lambda_k))$ computed over $(a,b)$ parameter space by solving \eqref{realfixbif} and \eqref{complexfixbif}, with $\epsilon^2=0.1$, $L^2=9/2$. As $\tau$ increases, $|\max_k(\Re(\lambda_k))|$ decreases. Contour lines for $\Re(\lambda_0)=0$ and $\max_k(\Re(\lambda_k))=0$ overlayed, indicated Turing instability region. $\max_k$ taken over $k\in[0, 50]$ at discrete intervals of $1$.}
    \label{fig:fixbif2}
\end{figure}
\section{Investigation of Variation in Initial and Boundary Conditions}
In this section, the robustness of the results obtained in \cite{gaffmonk} are examined under varying of initial conditions and boundary conditions. We first consider the sensitivity of pattern formation in the context of fixed time delay to varying initial conditions. Three different sets of initial conditions are considered. $\text{IC}_1$ corresponds to the initial conditions used in \cite{gaffmonk}. The functional form of $\text{IC}_1$ can be found in appendix \ref{section:appA}. $\text{IC}_2$ denotes the same initial conditions defined in \eqref{firstic}, and $\text{IC}_3$ are the initial conditions given by
\begin{equation}\label{ic3}
\text{IC}_3:\quad\quad\quad\begin{pmatrix}u_0\\v_0\end{pmatrix}=\begin{pmatrix}u_\star(1+r)\\v_\star(1+r)\end{pmatrix}\quad r\sim\mathcal{N}\left(0,0.1^2\right).
\end{equation}
We note that computationally a fixed random seed was set. The model parameters used match those used in \cite{gaffmonk}, with $(a,b)=(0.1,0.9)$. The results in figures \ref{fig:figtau0}, \ref{fig:figtau1}, \ref{fig:figtau2}, \ref{fig:figtau4}, and \ref{fig:figtau8} show the pattern formation observed for each of the initial conditions for varying fixed time delay $\tau\in\{0,1,2,4,8 \}$.

\begin{figure}[H]
    \centering
    \begin{subfigure}[b]{0.32\textwidth}
        \centering
        \includegraphics[width=5cm,height=4.5cm]{gaff0.png}
        \caption{$\text{IC}_1$ given in \cite{gaffmonk}.}
        \label{}
    \end{subfigure}
    \hfill
    \begin{subfigure}[b]{0.32\textwidth}
        \centering
        \includegraphics[width=5cm,height=4.5cm]{ic20.png}
        \caption{$\text{IC}_2$ given by \eqref{firstic}.}
        \label{}
    \end{subfigure}
    \hfill
    \begin{subfigure}[b]{0.32\textwidth}
        \centering
        \includegraphics[width=5cm,height=4.5cm]{ic30.png}
        \caption{$\text{IC}_3$ given by \eqref{ic3}.}
        \label{}
    \end{subfigure}
    \caption{Numerical simulations of \eqref{fixed2} showing comparison of varying ICs for $\tau=0$. Boundary conditions given by \eqref{neumannbc}. $(a,b)=(0.1,0.9)$, $\epsilon^2=0.001$, $L^2=9/2$. }
    \label{fig:figtau0}
\end{figure}
\begin{figure}[H]
    \centering
    \begin{subfigure}[b]{0.32\textwidth}
        \centering
        \includegraphics[width=5cm,height=4.5cm]{gaff1.png}
        \caption{$\text{IC}_1$ given in \cite{gaffmonk}.}
        \label{}
    \end{subfigure}
    \hfill
    \begin{subfigure}[b]{0.32\textwidth}
        \centering
        \includegraphics[width=5cm,height=4.5cm]{ic21.png}
        \caption{$\text{IC}_2$ given by \eqref{firstic}.}
        \label{}
    \end{subfigure}
    \hfill
    \begin{subfigure}[b]{0.32\textwidth}
        \centering
        \includegraphics[width=5cm,height=4.5cm]{ic31.png}
        \caption{$\text{IC}_3$ given by \eqref{ic3}.}
        \label{}
    \end{subfigure}
    \caption{Numerical simulations of \eqref{fixed2} showing comparison of varying ICs for $\tau=1$. Boundary conditions given by \eqref{neumannbc}. $(a,b)=(0.1,0.9)$, $\epsilon^2=0.001$, $L^2=9/2$.}
    \label{fig:figtau1}
\end{figure}
\begin{figure}[H]
    \centering
    \begin{subfigure}[b]{0.32\textwidth}
        \centering
        \includegraphics[width=5cm,height=4.5cm]{gaff2.png}
        \caption{$\text{IC}_1$ given in \cite{gaffmonk}.}
        \label{}
    \end{subfigure}
    \hfill
    \begin{subfigure}[b]{0.32\textwidth}
        \centering
        \includegraphics[width=5cm,height=4.5cm]{ic22.png}
        \caption{$\text{IC}_2$ given by \eqref{firstic}.}
        \label{}
    \end{subfigure}
    \hfill
    \begin{subfigure}[b]{0.32\textwidth}
        \centering
        \includegraphics[width=5cm,height=4.5cm]{ic32.png}
        \caption{$\text{IC}_3$ given by \eqref{ic3}.}
        \label{}
    \end{subfigure}
    \caption{Numerical simulations of \eqref{fixed2} showing comparison of varying ICs for $\tau=2$. Boundary conditions given by \eqref{neumannbc}. $(a,b)=(0.1,0.9)$, $\epsilon^2=0.001$, $L^2=9/2$.}
    \label{fig:figtau2}
\end{figure}
\begin{figure}[H]
    \centering
    \begin{subfigure}[b]{0.32\textwidth}
        \centering
        \includegraphics[width=5cm,height=4.5cm]{gaff4.png}
        \caption{$\text{IC}_1$ given in \cite{gaffmonk}.}
        \label{}
    \end{subfigure}
    \hfill
    \begin{subfigure}[b]{0.32\textwidth}
        \centering
        \includegraphics[width=5cm,height=4.5cm]{ic24.png}
        \caption{$\text{IC}_2$ given by \eqref{firstic}.}
        \label{}
    \end{subfigure}
    \hfill
    \begin{subfigure}[b]{0.32\textwidth}
        \centering
        \includegraphics[width=5cm,height=4.5cm]{ic34.png}
        \caption{$\text{IC}_3$ given by \eqref{ic3}.}
        \label{}
    \end{subfigure}
    \caption{Numerical simulations of \eqref{fixed2} showing comparison of varying ICs for $\tau=4$. Boundary conditions given by \eqref{neumannbc}. $(a,b)=(0.1,0.9)$, $\epsilon^2=0.001$, $L^2=9/2$.}
    \label{fig:figtau4}
\end{figure}
\begin{figure}[H]
    \centering
    \begin{subfigure}[b]{0.32\textwidth}
        \centering
        \includegraphics[width=5cm,height=4.5cm]{gaff8.png}
        \caption{$\text{IC}_1$ given in \cite{gaffmonk}.}
        \label{}
    \end{subfigure}
    \hfill
    \begin{subfigure}[b]{0.32\textwidth}
        \centering
        \includegraphics[width=5cm,height=4.5cm]{ic28.png}
        \caption{$\text{IC}_2$ given by \eqref{firstic}.}
        \label{}
    \end{subfigure}
    \hfill
    \begin{subfigure}[b]{0.32\textwidth}
        \centering
        \includegraphics[width=5cm,height=4.5cm]{ic38.png}
        \caption{$\text{IC}_3$ given by \eqref{ic3}.}
        \label{}
    \end{subfigure}
    \caption{Numerical simulations of \eqref{fixed2} showing comparison of varying ICs for $\tau=8$. Boundary conditions given by \eqref{neumannbc}. $(a,b)=(0.1,0.9)$, $\epsilon^2=0.001$, $L^2=9/2$.}
    \label{fig:figtau8}
\end{figure}

It can be seen that the final pattern is sensitive to the choice of initial conditions, and that, intuitively, the larger $\sigma_{\text{IC}}$ used in $\text{IC}_3$, compared to that of $\text{IC}_2$ results in a faster onset of pattern formation. We see from considering the timescales as to which pattern formation occurs however, that although the time taken until onset of patterning varies with different initial conditions, the increase in time-to-pattern with an increasing time delay is consistent independent of the initial conditions chosen. By considering the varying $x$-axis, we also note that in each case, this relationship seems to be linear. We formalise this in section \ref{section:delaypatt}.

Numerical results were also simulated to study the effects of a temporal variation in the history function. A history function was set as $h(t)=u_\star(1+r\sin(\omega t))$ for $t\in[-\tau,0)$, where $r$ is the random variable used in $\text{IC}_2$. Simulations were conducted for varying $\tau$ and $\omega$. Preliminary simulations, which can be found in appendix \ref{section:appB}, show that this type of variation in history does not have a significant effect on the results seen.

Finally, we consider the effect of varying boundary conditions. Motivated from the analysis in \cite{krausemixed}, homogeneous Dirichlet boundary conditions are implemented for the activator term, and homogeneous Neumann boundary conditions implemented for the inhibitor term. Thus, we have that, on a domain $\Omega=[0,1]$,
\begin{equation}\label{homogeneousbc}
u=\frac{\partial v}{\partial t}=0 \quad x=0, 1.
\end{equation}
These conditions are implemented numerically following the methodology outlined in section \ref{section:numimp}. The results in figures \ref{fig:bctau1}, \ref{fig:bctau2}, and \ref{fig:bctau3} were generated using $\text{IC}_2$, with a varying $\tau\in\{0,1,8\}$, and show the comparison between numerical simulations generated with homogeneous Neumann conditions for both $u$ and $v$, as in \eqref{neumannbc} indicated as $\text{BC}_1$, and those generated with homogeneous Dirichlet conditions for $u$, indicated as $\text{BC}_2$, as in \eqref{homogeneousbc}

\begin{figure}[H]
    \centering
    \begin{subfigure}[b]{0.45\textwidth}
        \centering
        \includegraphics[width=7cm,height=5.5cm]{ic20.png}
        \caption{$\text{BC}_1$ given by \eqref{neumannbc}.}
        \label{}
    \end{subfigure}
    \hfill
    \begin{subfigure}[b]{0.45\textwidth}
        \centering
        \includegraphics[width=7cm,height=5.5cm]{bc0.png}
        \caption{$\text{BC}_2$ given by \eqref{homogeneousbc}.}
        \label{}
    \end{subfigure}
    \caption{Comparison of varying BCs for $\tau=0$. Generated with $\text{IC}_2$. $(a,b)=(0.1,0.9)$, $\epsilon^2=0.001$, $L^2=9/2$. Initial conditions given by \eqref{firstic}.}
    \label{fig:bctau1}
\end{figure}

\begin{figure}[H]
    \centering
    \begin{subfigure}[b]{0.45\textwidth}
        \centering
        \includegraphics[width=7cm,height=5.5cm]{ic21.png}
        \caption{$\text{BC}_1$ given by \eqref{neumannbc}.}
        \label{}
    \end{subfigure}
    \hfill
    \begin{subfigure}[b]{0.45\textwidth}
        \centering
        \includegraphics[width=7cm,height=5.5cm]{bc1.png}
        \caption{$\text{BC}_2$ given by \eqref{homogeneousbc}.}
        \label{}
    \end{subfigure}
    \caption{Comparison of varying BCs for $\tau=1$. Generated with $\text{IC}_2$. $(a,b)=(0.1,0.9)$, $\epsilon^2=0.001$, $L^2=9/2$. Initial conditions given by \eqref{firstic}.}
    \label{fig:bctau2}
\end{figure}

\begin{figure}[H]
    \centering
    \begin{subfigure}[b]{0.45\textwidth}
        \centering
        \includegraphics[width=7cm,height=5.5cm]{ic28.png}
        \caption{$\text{BC}_1$ given by \eqref{neumannbc}.}
        \label{}
    \end{subfigure}
    \hfill
    \begin{subfigure}[b]{0.45\textwidth}
        \centering
        \includegraphics[width=7cm,height=5.5cm]{bc8.png}
        \caption{$\text{BC}_2$ given by \eqref{homogeneousbc}.}
        \label{}
    \end{subfigure}
    \caption{Comparison of varying BCs for $\tau=8$. Generated with $\text{IC}_2$. $(a,b)=(0.1,0.9)$, $\epsilon^2=0.001$, $L^2=9/2$. Initial conditions given by \eqref{firstic}.}
    \label{fig:bctau3}
\end{figure}

We note that, although changing the boundary conditions for the activator term $u$ to homogeneous Dirichlet conditions affects the type of patterns we may see (number and amplitude of spikes), this change does not affect the increased timescales, caused by an increase in time delay, on which onset of patterning occurs.

\section{Relationship Between Time-To-Pattern and time delay}\label{section:delaypatt}

We aim to show that, for small $\tau$ and small $L$, the linear theory provides a good approximation to the time taken until pattern formation occurs, and in fact, the relationship between $\tau$ and time-to-pattern under these conditions is linear. We also show that, through full numerical solutions, the relationship between $\tau$ and time-to-pattern on a longer time-scale for larger $\tau$ is also linear. We first consider the former.

To minimise the effect of nonlinearity in the dynamics, we restrict the domain size to $L^2=1/5$. Shrinking the domain results in fewer unstable modes and thus less competition for the dominant mode, resulting in a better approximation of the linear theory. This finite size effect can be seen in figure \ref{fig:compardisp}, where $\Re(\lambda_k)$ is plotted against $k$ for two different domain sizes, for a given $(a,b,\tau)$. Due to numerical restrictions when using Chebfun in finding roots of the characteristic equation \eqref{characfix}, only $\tau\leq1.6$ is considered. Similar to the initial conditions used for previous numerical simulations (figure \ref{fig:fixedsim2}), we take a small perturbation in the activator term, $\hat{u}(t)$, such that $\hat{u}(0)=ru_\star$, where $r$ is a small Gaussian random variable, $r\sim\mathcal{N}\left(0,\sigma_{\text{IC}}^2\right)$,
for some standard deviation of the initial perturbation $\sigma_{\text{IC}}$.

The linear theory suggests that at some time $t=T$, the perturbation will be of the form $\hat{u}(T)\sim A_k(T)\cos(k\pi x)$, where $k$ is the dominant mode and $A_k(T)$ denotes the corresponding Fourier coefficient at time $t=T$. For a given parameter set $(a,b,\epsilon^2,\tau,L)$, we can solve the characteristic equation \eqref{characfix}, and plot $\Re(\lambda_k)$
against $k$, to determine the dominating mode $k$ and the corresponding eigenvalue, or growth rate, $\lambda_k$. We then use this information in the following manner: A Fast Fourier Transform to decompose the initial conditions into a Fourier series is used, and the coefficient $A_k(0)$ for the dominating $k$ is computed. When the perturbation $\hat{u}$ has grown sufficiently, in absolute value, beyond a threshold where pattern formation is considered, we call this time $t=T$, and again determine the Fourier coefficient $A_k(T)$ of the fastest-growing mode $k$. More specifically, the time $T$ is the first such that $\max_x|u(T,x)-u_\star|>threshold$, namely the first time such that any solution point across the whole spatial domain is large enough, in absolute difference, from the steady-state. Finally, using the relation $A_k(T)\sim A_k(0)e^{\lambda_k T}$, we rearrange for $T$ and thus compute a linear approximation for time-to-pattern as
\begin{equation}\label{ttprelation}
    T=\frac{1}{\lambda_k}\ln\left(\frac{A_k(T)}{A_k(0)}\right).
\end{equation}
We consider an example case for $(a,b,\tau)=(0.4,1.8,0.2)$. The standard deviation for the random variable $r$ is chosen as $\sigma_{\text{IC}}=10^{-5}$, and the threshold value at $0.1$. A very small perturbation was used as a means to improve the accuracy of the linear theory.
\begin{figure}[H]
    \centering
    \begin{subfigure}[b]{0.45\textwidth}
        \centering
        \includegraphics[width=7cm,height=5.5cm]{compdisp1.png}
        \caption{Dispersion curve plotted with domain size $L^2=1/5$. Curve produced by varying $k\in[0,5]$ at regular intervals of $0.1$. Discrete values of $k$ overlayed as scatter points. }
        \label{fig:compdisp1}
    \end{subfigure}
    \hfill
    \begin{subfigure}[b]{0.45\textwidth}
        \centering
        \includegraphics[width=7cm,height=5.5cm]{compdisp2.png}
        \caption{Dispersion curve plotted with domain size $L^2=9/2$. Curve produced by varying $k\in[0,60]$ at regular intervals of $1$.}
        \label{fig:compdisp2}
    \end{subfigure}
    \caption{Dispersion curves of the characterstic equation given in \eqref{characfic} plotted for $(a,b,\tau)=(0.4,1.8,0.2)$ and $\epsilon^2=0.001$. A larger $L$ results in more unstable modes $\lambda_k$ such that $\Re(\lambda_k)>0$. }
    \label{fig:compardisp}
\end{figure}
Using $\epsilon^2=0.001$, on the domain size $L=2\sqrt{0.05}$, figure \ref{fig:compdisp1} suggests that, from the linear theory, the dominant mode is $k=1$ with dominant eigenvalue $\lambda_1=0.2356$. Since $k=1$ is the dominant mode, we compute the first Fourier coefficient of the initial conditions, $A_1(0)$, as $A_1(0)=7.95(3.s.f)\times10^{-8}$. To find $A_1(T)$, a numerical simulation is run until the solution of the activator $u$ has grown, in absolute value, to a threshold value of $0.1$. Figure \ref{fig:Tfc} shows the numerical solution $u(T)$ at the point where this threshold value has been met, as well as a scatter plot of the Fourier coefficients $A_k(T)$, $k\neq0$.
\begin{figure}[H]
    \centering
    \begin{subfigure}[b]{0.45\textwidth}
        \centering
        \includegraphics[width=7cm,height=6cm]{Tu.png}
        \caption{Numerical solution $u(T)$ at $t=T$}
        \label{uT}
    \end{subfigure}
    \hfill
    \begin{subfigure}[b]{0.45\textwidth}
        \centering
        \includegraphics[width=7cm,height=6cm]{FCs.png}
        \caption{Absolute Fourier coefficients of $u(T)$, for $k\in[1,50]$.}
        \label{fig:uTfc}
    \end{subfigure}
    \caption{Numerical solution of \eqref{fixed2} at $t=T$ with homogeneous Neumann boundary conditions implemented. Initial perturbation from steady-state with $\sigma_{\text{IC}}=10^{-5}$. First $50$ Fourier coefficients for $u(T)$ plotted, with $(a,b)=(0.4,1.8)$, time delay $\tau=0.2$ and $\epsilon^2=0.001$, $L=2\sqrt{0.05}$.}
    \label{fig:Tfc}
\end{figure}
As seen in figure \ref{fig:uTfc}, the Fourier coefficient corresponding to $k=1$ is given as $0.0262(3.s.f)$. The approximated time-to-pattern, as predicted by linear theory, for $(a,b,\tau)=(0.4,1.8,0.2)$ and the given initial conditions is thus computed as
\begin{equation}
    T=\frac{1}{\lambda_1}\ln\left(\frac{A_1(T)}{A_1(0)}\right)=\frac{1}{0.2356}\ln\left(\frac{0.0262}{7.95\times10^{-8}}\right)=53.8(3.s.f).
\end{equation}
It was found through numerical solutions that the `true' time-to-pattern is $\approx57.5(3sf)$.
We use `true' time-to-pattern here to mean the time taken for a perturbation to grow above a threshold value found through full numerical solutions. This process can be repeated for varying $(a,b,\tau)$, and figures \ref{fig:ttp1}, \ref{fig:ttp2}, \ref{fig:ttp3} show the predicted time-to-pattern plotted against $\tau$ and compared with the `true' time-to-pattern for three different parameter sets. The time delay is varied here over $\tau\in[0,1.6]$ at intervals of $0.2$.

\begin{figure}[H]
    \centering
    \begin{subfigure}[b]{0.32\textwidth}
        \centering
        \includegraphics[width=5cm,height=5cm]{ttp1.png}
        \caption{$(a,b)=(0.4,1.8)$.}
        \label{fig:ttp1}
    \end{subfigure}
    \hfill
    \begin{subfigure}[b]{0.32\textwidth}
        \centering
        \includegraphics[width=5cm,height=5cm]{ttp2.png}
        \caption{$(a,b)=(0.1,0.9)$.}
        \label{fig:ttp2}
    \end{subfigure}
    \hfill
    \begin{subfigure}[b]{0.32\textwidth}
        \centering
        \includegraphics[width=5cm,height=5cm]{ttp3.png}
        \caption{$(a,b)=(0.2,1.3)$.}
        \label{fig:ttp3}
    \end{subfigure}
    \caption{Predicted vs `true' time-to-pattern for numerical solution of \eqref{fixed2} with homogeneous Neumann boundary conditions implemented. Initial perturbation from steady-state with $\sigma_{\text{IC}}= 10^{-5}$. Predicted time-to-pattern computed using the relationship \eqref{ttprelation}, for three different parameter sets, with $L=2\sqrt{0.05}$, $\epsilon^2=0.001$, and $\tau\in[0,1.6]$.}
    \label{}
\end{figure}

Now, through full numerical solutions, we show a linear relationship between $\tau$ and time-to-pattern on a longer time-scale. Varying $\tau\in[1,16]$ at regular intervals of $1$, for two different parameter sets $(a,b)=\{(0.1,0.9),(0.4,0.8)\}$, we compute the time taken for a perturbation to grow up to a threshold value, and plot the results. Figure \ref{fig:linperturb1a} shows the results for $(a,b)=(0.1,0.9)$ with an initial perturbation of standard-deviation $\sigma_{\text{IC}}=10^{-5}$ and threshold value $0.1$. Figure \ref{fig:linperturb1b} shows the results for the same parameter values but with an initial perturbation of standard-deviation $\sigma_{\text{IC}}=0.01$ and threshold value $2$. Figures \ref{fig:linperturb2a} and \ref{fig:linperturb2b} show analogous results but for $(a,b)=(0.4,0.8)$. The simulations were run with $\epsilon^2=0.001$ and $L^2=9/2$.
\begin{figure}[H]
    \centering
    \begin{subfigure}[b]{0.45\textwidth}
        \centering
        \includegraphics[width=7cm,height=6cm]{longlin2.png}
        \caption{$\sigma_{\text{IC}}=10^{-5}$ and threshold $0.1$.}
        \label{fig:linperturb1a}
    \end{subfigure}
    \hfill
    \begin{subfigure}[b]{0.45\textwidth}
        \centering
        \includegraphics[width=7cm,height=6cm]{longlin1.png}
        \caption{$\sigma_{\text{IC}}=0.01$ and threshold $2$}
        \label{fig:linperturb1b}
    \end{subfigure}
    \caption{Time-to-pattern for full numerical solutions of \eqref{fixed2} plotted against $\tau\in[1,16]$ for two different initial perturbations and threshold values. $(a,b)=(0.1,0.9)$, $\epsilon^2=0.001$ and domain size $L^2=9/2$.}
    \label{fig:linperturb1}
\end{figure}

\begin{figure}[H]
    \centering
    \begin{subfigure}[b]{0.45\textwidth}
        \centering
        \includegraphics[width=7cm,height=6cm]{longlin3.png}
        \caption{$\sigma_{\text{IC}}=10^{-5}$ and threshold $0.1$}
        \label{fig:linperturb2a}
    \end{subfigure}
    \hfill
    \begin{subfigure}[b]{0.45\textwidth}
        \centering
        \includegraphics[width=7cm,height=6cm]{longlin4.png}
        \caption{$\sigma_{\text{IC}}=0.01$ and threshold $2$}
        \label{fig:linperturb2b}
    \end{subfigure}
    \caption{Time-to-pattern for full numerical solutions of \eqref{fixed2} plotted against $\tau\in[1,16]$ for two different initial perturbations and threshold values. $(a,b)=(0.4,0.8)$, $\epsilon^2=0.001$ and domain size $L^2=9/2$.}
    \label{fig:linperturb2}
\end{figure}


\section{Summary}

In this chapter, we presented the biological motivation for studying the LI variant of the Schnakenberg model with incorporated fixed delay.

Linear theory suggested that for the LI model, time delay can act as a stabilising agent for Turing instabilities, increasing the parameter region where Turing instabilities can occur. Through both linear analysis on a small scale, and full numerical solutions on a larger scale, a linear relationship between time delay and time-to-pattern was presented, and an increasing delay was shown to, depending on the parameter chosen, increase the time taken for pattern formation to occur, or increase the time taken for perturbations to fully decay.

Numerical results in this chapter were also systematically tested with a varying of initial and boundary conditions, and a temporal variation in the history function. Simulations suggested that the increase in time-to-pattern with an increase in delay is robust to these variations. We therefore look for ways to remedy the problems caused by a fixed delay. Considering the complexity and stochastic nature of pattern formation on a cellular level leads us to consider modelling the time delay as a distrbution, which we consider in the next chapter.
