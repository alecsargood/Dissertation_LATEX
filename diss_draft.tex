\documentclass[12pt,a4paper]{report}

\setlength{\topmargin}{0.0in}
\setlength{\oddsidemargin}{0.33in}
\setlength{\textheight}{9.0in}
\setlength{\textwidth}{6.0in}
\renewcommand{\baselinestretch}{1.25}

\usepackage{nicematrix}
\usepackage[margin=0.75in]{geometry}
\usepackage[edges]{forest}
\usepackage{graphicx,wrapfig,lipsum}
\usepackage{amsmath,empheq}
\usepackage{dirtree}
\usepackage{titlesec}
\setcounter{secnumdepth}{4}
\titleformat{\paragraph}
{\normalfont\normalsize\bfseries}{\theparagraph}{1em}{}
\titlespacing*{\paragraph}
{0pt}{3.25ex plus 1ex minus .2ex}{1.5ex plus .2ex}
\usepackage{amssymb}
\usepackage{amsthm}
\usepackage[style=numeric,backend=biber]{biblatex}
\addbibresource{bibliographydiss.bib}
\usepackage{tikz}
\def\checkmark{\tikz\fill[scale=0.4](0,.35) -- (.25,0) -- (1,.7) -- (.25,.15) -- cycle;}
\usepackage[utf8]{inputenc}
\usepackage[english]{babel}
\usepackage{hyperref}
\usepackage[ruled]{algorithm2e}
\usepackage{amssymb}
\usepackage{graphicx}
\usepackage{cancel}
\usepackage{epstopdf}
\usepackage{mathtools}
\usepackage{float}
\usepackage{caption}
\usepackage{subcaption}
\usepackage{setspace}
\usepackage{bm}
\usepackage[export]{adjustbox}
\usepackage{relsize}
\usepackage{listings}
\usepackage[toc,page]{appendix}
\usepackage{cancel}

\title{Effects of Distributed Delay on Turing Pattern Formation in Reaction-Diffusion Systems.}
\author{Alec Sargood}
\begin{document}
\date{2021}
\maketitle

\tableofcontents
\newpage
\chapter{Introduction}

\section{Background}

Morphogenesis is a complex spatio-temporal process responsible for the development of cellular shape or form, for example, the controlling the spatial distribution of cells during the development process of an embryo, or the growth of cancerous tissue. (REFERENCE) In 1952 \cite{turing}, Turing proposed that the morphogenesis process could be mathematically modelled as a purely chemical basis via the interaction of morphogens, or reactants, whose evolutions are described by a system of coupled reaction-diffusion equations. Turing's models have been used to explain the apparent self-organisation of cells leading to naturally occurring pattern formation, such as on animal coats (e.g. spots on a jaguar \cite{painter}), the formation of features (e.g. pattern of feathers on birds \cite{bailleul}), and in developmental biology (e.g. vertebrate limb development \cite{miura1,glimm,miura2}). It was proposed that a stable steady-state, robust to small perturbations in the spatially homogeneous setting, could become unstable and sensitive to small perturbations with the introduction of diffusion, leading to spatially inhomogeneous patterns. Physically, the spatially homogeneous state can be thought of as representing early stages of development, e.g a zygote with negligible domain size. The diffusion driven instability therefore can be considered as a result of the embryogenesis process, where the zygote undergoes cellular divisions (REFERENCE), leading to the embryo taking a specific shape and form.
\\
Morphogenesis relies on the ability of cells to adopt a state appropriate to their temporal and spatial position \cite{gaffmonk}. The mechanism allowing cells to adopt an appropriate state is known as differential gene expression, and depends tightly on the communication between cells, achieved through cellular signalling. Typical reaction-diffusion systems assume a neglibible timescale on which the cellular signalling and gene expression processes occur. The gene expression process is however extremely complex and proceeds through several stages \cite{gaffmonk}, namely gene transcription and gene translation. These sub-processes can take large amounts of time, and it has been experimentally shown that these time delays are typically on the order of minutes, but in some cases can be as large as a few hours \cite{gaffmonk,tennyson}. These time delays can therefore be on the same order of magnitude as the pattern formation process itself. For example, the basic body plan of a zebrafish is established in less than 24 hours \cite{gaffmonk,kimmel}. It is therefore important to consider these delays when studying pattern formation in Turing mechanisms.
\\
Such time delays have been investigated in the context of Turing patterns \cite{gaffmonk,leegaffney,yigaffneyli,jiang,leegaffmonk,bratsun,william}, by incorporating fixed constant delays into the models. The numerical results in \cite{gaffmonk}, whose model description we will use, showed that the time taken until pattern formation occurs drastically increases as the gene expression time delay in the model is increased, and that small delays can cause a large increase in time-to-pattern. Numerical simulations on a growing domain also suggested that time delay can cause a failur in Turing pattern formation. Similar results have also been discussed in \cite{leegaffney,leegaffmonk}, where simulations have shown the disruption time delay can cause to pattern formation, as well as inducing temporal oscillations and increasing sensitivity of Turing instabilities to initial conditions. An analytical study of time delay in \cite{yigaffneyli} has shown that for certain reaction-diffusion systems and certain delay terms, increasing time delay can antagonise the the induction of Turing patterns. It has in general been found that these time delays can increase the time taken until pattern formation occurs, induce oscillations, or completely prevent pattern formation, and therefore pose a potential obstacle to applying Turing models to real systems. In reality, the stochastic nature of gene transcription and translation means that the fixed constant delay is an oversimplification of the underlying cellular process, and leads us to consider a distribution of time delays at the macroscopic level \cite{bratsun}.
\\
In this report we are interested in whether implementing different forms of the delay, namely distributed delay, can alleviate some of the problems caused by the fixed delay case. We will first outline some of the mathematical theory underpinning Turing pattern formation as well as introducing the mathematical models we will be studying. Chapter 2 will be concerned with verifying and extending the current literature concerned with the fixed delay model. We will specifically be concerned with evaluating the robustness of current results to varying initial and boundary conditions. Finally, Chapter 3 will focus on the distributed delay model. We will develop the numerics required to study the model, and present our results and analysis.

\section{Model Introduction}
The mathematical model we will consider is the Schnakenberg model \cite{schnakenberg} - one of the simplest `toy' models that exhibit some of the key behaviours that we are interested in, as well as one that has been studied extensively in the context of fixed time delays. The model describes the evolution and interaction between two reactants, $u$ and $v$. Only considering two reactants is a gross simplification of the biological process on a cellular level, but it is still a non-trivial case than can admit Turing patterns. In this report, we restrict our investigation to one spatial domain. The non-dimensionalised model description we use is motivated from \cite{gaffmonk} and is given by

\begin{equation}\label{system}
    \begin{split}
    \frac{\partial u}{\partial t}&=\frac{\epsilon^2}{\gamma}\frac{\partial^2 u}{\partial x^2}+f(u,v),\\
    \frac{\partial v}{\partial t}&=\frac{1}{\gamma}\frac{\partial^2 v}{\partial x^2}+g(u,v) \quad\quad\quad x\in\Omega,
    \end{split}
\end{equation}
where $\Omega$ is the spatial domain, which we consider here as $\Omega=[0,L]$, for some domain length $L\in\mathbb{R}_{>0}$. Since we are interested in the pattern formation arising from the self-organisation of cells, we implement no flux (homogeneous Neumann) boundary conditions on the boundary of the spatial domain $\partial\Omega$, namely $\frac{\partial u}{\partial x}=\frac{\partial v}{\partial x}=0$ at $x=0, L$. We also assume initial conditions $u(x,0)$ and $v(x,0)$ are given. The functions $f$ and $g$ are the functions describing the intrinsic chemical reaction between the two morphogens $u$ and $v$, and for the Schnakenberg model are given as
\begin{equation}
    \begin{split}
        f(u,v)&=a-u+u^2v\\
        g(u,v)&=b-u^2v,
    \end{split}
\end{equation}
for parameters $a$ and $b$.
As typical when studying Turing patterns, initial conditions will be chosen as a small random perturbation from the spatially homogeneous steady state, found as the roots of $f$ and $g$.
Further details on the model parameters $a,b,\epsilon,\gamma$ can be found in \cite{gaffmonk}, and unless otherwise stated, we use $\gamma=5.0\times10^{-2}$ and $\epsilon^2=0.001$. Importantly, $\epsilon^2$ is the ratio of diffusion coefficients where usually $\epsilon\ll1$, and $a$ and $b$ are the reaction kinetic parameters. Necessary conditions on $(a,b,\epsilon^2)$ can also be found for Turing pattern formation to occur.



\section{Turing Pattern Formation Without Delay}

Here we give a brief overview of the mathematical theory underpinning Turing's mechanism, closely following the description in \cite{murray}. For further details, the reader should consult \cite{murray,beentjes}. Turing patterns occur when the spatially homogeneous stable steady-state becomes unstable in the presence of diffusion. We therefore first consider the spatially homogeneous model, and explore conditions necessary for the steady-state to be stable. In the case of the Schnakenberg model, the single steady-state, $(u_\star, v_\star)$, occurs at $(u_\star, v_\star)=\left(a+b, \frac{b}{(a+b)^2}\right)$. Since $a,b>0$, we have $u_\star,v_\star>0$ which is physically feasible. Following the methodology in \cite{murray}, we perform linear stability analysis. Taking a small perturbation from the steady-state, so that $\hat{u}=u-u_\star$, $\hat{v}=v-v_\star$ for $|\hat{u}|, |\hat{v}|<<1$, we consider the evolution of the perturbation. Denoting $\hat{\textbf{u}}$ as the vector of perturbations, $\hat{\textbf{u}}=\begin{bmatrix}\hat{u} \\ \hat{v}\end{bmatrix}$, the linearised system of \eqref{system} is given as
\begin{equation}\label{linsys}
\frac{\text{d}\hat{\textbf{u}}}{\text{dt}}=J_{(u_\star,v_\star)}\hat{\textbf{u}},
\end{equation}
where $J_{(u_\star,v_\star)}$ is the Jacobian matrix of the kinetic equations evaluated at the steady-state. Namely,
$$
J=\begin{pmatrix}f_u&f_v\\g_u&g_v\end{pmatrix}.
$$
The notation $f_u$ is used to denote the partial derivatve of $f$ with respect to $u$. Solutions of \eqref{linsys} are of the form
$$
\hat{\textbf{u}}\propto e^{\lambda t},
$$
for eigenvalues $\lambda$ of $J_{(u_\star,v_\star)}$. The steady-state is stable if the perturbation decays. This occurs when $Re(\lambda)<0 \ \text{for all }   \lambda\in \text{spec}(J_{(u_\star,v_\star)})$, namely when the real part of all the eigenvalues of the Jacobian matrix, evaluated at the steady-state, are negative. However, if there exists $\lambda\in \text{spec}(J_{(u_\star,v_\star)}): Re(\lambda)>0$, then the pertubation will grow with time and the steady-state in unstable. The sum and product of the eigenvalues of $J_{(u_\star,v_\star)}$ are given by $\text{Tr}(J_{(u_\star,v_\star)})$ and $\text{det}(J_{(u_\star,v_\star)})$ respectively. The required conditions for stability are therefore
\begin{equation}\label{cond1}
    \begin{split}
\text{Tr}(J_{(u_\star,v_\star)})<0 &\implies f_u+g_v<0, \\
\text{det}(J_{(u_\star,v_\star)})>0 &\implies f_ug_v-f_vg_u>0,
\end{split}
\end{equation}
where the derivatives are evaluated at the steady-state $(u_\star,v_\star)$.
We now consider the full diffusive model and look for necessary conditions such that the previously stable steady-state is driven to instability. The linearised system is given by
\begin{equation}\label{linsys2}
    \frac{\partial \hat{\textbf{u}}}{\partial t}=\left[\textbf{D}\nabla^2+J_{(u_\star,v_\star)} \right]\hat{\textbf{u}},
\end{equation}
where $\textbf{D}=\begin{pmatrix}\epsilon^2/\gamma&0\\0&1/\gamma\end{pmatrix}$ is the matrix containing the diffusion coefficients of reactants.
The solution to the spatially dependent eigenvalue problem can be written as a linear combination of the eigenfunctions $w_k$ that satisfy the problem
\begin{equation}\label{eigprob}
\nabla^2w_k=-k^2w_k,\quad \quad \frac{\partial w_k}{\partial \textbf{n}}=0\text{ for } \textbf{x}\in\partial\Omega.
\end{equation}
Equation \eqref{eigprob} is valid since we consider here only a regular 1D domain $\Omega=[0,L]$ for some domain length $L\in\mathbb{R}_{>0}$. $\partial\Omega$ denotes the boundary of the domain, and $n$ denotes the outward-facing normal.
We thus look for solutions to \eqref{linsys2} of the form
\begin{equation}
    \hat{\textbf{u}}=\sum_k \textbf{c}_ke^{\lambda_k t}w_k(\textbf{x}),
\end{equation}
where the constants $\textbf{c}_k$ are determined by using a Fourier expansion of the initial conditions in terms of the eigenfunctions $w_k$. $\lambda_k$ is the eigenvalue which determines the rate of temporal growth for each mode $k$, and thus determines whether a particular mode of pattern will be unstable and grow. Substituting this form into \eqref{linsys2}, along with using \eqref{eigprob} and simplifying, we obtain
$$
\lambda w_k=Jw_k-\textbf{D}k^2w_k \implies (\lambda I-J+k^2\textbf{D})w_k=\textbf{0}.
$$
Looking for non-trivial solutions for $w_k$, we set solve for roots of the characteristic polynomial, namely $\text{det}(\lambda I-J+k^2\textbf{D})=0$, which yields a quadratic equation for eigenvalues $\lambda(k)$ as a function of $k$. Finding roots of this quadratic such that $Re(\lambda(k))>0$ for some $k\neq0$, we can conclude two necessary conditions for the instability of the steady-state in the presence of diffusion, namely
\begin{equation}\label{cond2}
    \begin{split}
    \frac{1}{\epsilon^2}f_u+g_v>0&\\
    \left(\frac{1}{\epsilon^2}f_u+g_v\right)^2-\frac{4}{\epsilon^2}(f_ug_v-f_vg_u)>0.
\end{split}
\end{equation}
We therefore have four necessary conditions in terms of $(a,b,\epsilon^2)$ for Turing patterns to occur. Using the first two conditions in \eqref{cond1}, a bifurcation diagram in the $(a,b)$ parameter space can be plotted showing the regions corresponding to a stable or unstable steady state. This can be seen in figure (FIGURE). Using the additional conditions in \eqref{cond2} and the fixed value $\epsilon^2=0.001$, the parameter region in the $(a,b)$ parameter space in which Turing patterns can occur can also be plotted. This `Turing space' can be seen in (FIGURE).
\\
\\
\\

(INCLUDE TWO FIGURES HERE).


\printbibliography

\end{document}
